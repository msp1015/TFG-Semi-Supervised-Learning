\apendice{Plan de Proyecto Software}

\section{Introducción}

\section{Planificación temporal}
Como se explica en la sección 4 de la docucmentación, la metodología a seguir es la de scrum, salvando las distancias con el número de personas que desarrollan.
El proyecto se elige antes de empezar el curso, unicamente para poder informarse y leer articulos relacionados con el tema. Empezando por \cite{Engelen:semi-supervised} e investigando los trabajos anteriores de otros compañeros.
\subsection{Sprint 1}
Este sprint corresponde a las fechas entre el 7 de noviembre y el 18 de diciembre. Se comenzó con una reunion en la que se establecieron las siguientes tareas:\\
Crear un repositorio en GitHub donde poder subir los cambios del proyecto, más concretamente, la plantilla de documentación. También se mandó terminar de leer el artículo \cite{Engelen:semi-supervised} y se asignó una nueva lectura acerca de ensembles, \cite{ensembles}.
Tambien se asigna la tarea de encontrar un programa util para el seguimiento del proyecto que soporte scrum, seleccionando finalmente \textit{Taiga}.
\subsection{Sprint 2}
Sprint correspondiente a las fechas entre el 18 de noviembre y el 15 de enero. Se inicia con una reunión previa para establecer las tareas: 
finalizar la lectura de \cite{ensembles}, comenzar la documentación con conceptos teoricos acerca del aprendizaje automático vistos hasta el momento. Ponerse en contexto con el entorno de flask en python, que con ayuda de otras asignaturas se termina cumpliendo.
\subsection{Sprint 3}
Sprint correspondiente a las fechas entre el 15 de enero y el 1 de febrero. Se tiene una reunion previa para asignar las tareas de: lectura del algoritmo Co-Forest, busqueda de trabajos relacionados para coger ideas propias para el proyecto y continuar documentando los conceptos teoricos (ensembles, Co-Forest).

\subsection{Sprint 4}


\section{Estudio de viabilidad}

\subsection{Viabilidad económica}

\subsection{Viabilidad legal}

