\capitulo{4}{Técnicas y herramientas}

En este apartado se tratará de presentar las técnicas llevadas a cabo para desarrollar el proyecto y también las herramientas utilizadas durante todo el proceso.

\section{SCRUM}
Para explicar esta sección se utilizará el manual de la certificación oficial Scrum Master de Scrum Manager, ~\cite{SCRUM}.

\imagen{../img/memoria/cicloSCRUM.png}{Procedimiento en la metodología SCRUM ~\cite{SCRUM}}{1.1}.

La metodología ágil Scrum se diferencia de las metodologías clásicas en su enfoque iterativo y flexible hacia la gestión de proyectos. Mientras que las metodologías clásicas, como el modelo en cascada, siguen un enfoque secuencial y predeterminado, Scrum promueve la adaptabilidad y la colaboración continua. Facilita respuestas rápidas a los cambios a través de ciclos cortos de desarrollo llamados Sprints, permitiendo a los equipos evaluar el progreso y ajustar el rumbo con frecuencia.
\subsection{Roles}
\begin{itemize}
	\item \textbf{Desarrolladores}: Encargados de crear el producto, los desarrolladores son fundamentales para la ejecución técnica del proyecto, aportando habilidades específicas para alcanzar los objetivos del Sprint.
	\item \textbf{Propietario del producto(\textit{Product Owner})}: Define el alcance y las prioridades del proyecto, manteniendo la pila del producto actualizada para reflejar las necesidades del negocio, asegurando que el trabajo del equipo de desarrollo aporte el máximo valor.
	\item \textbf{\textit{Scrum Master}}: Facilitador y guía del equipo Scrum, el Scrum Master ayuda a implementar Scrum, asegurándose de que se sigan las prácticas y procesos, y trabaja para eliminar obstáculos que puedan impedir el progreso del equipo.
\end{itemize}
\subsection{Artefactos}
\begin{itemize}
	\item \textbf{Pila del producto(\textit{Product Backlog})}: Lista ordenada de todo lo necesario para el producto, gestionada por el \textit{Product Owner}, que establece los requisitos y prioridades.
	\item \textbf{Pila del sprint(\textit{Sprint Backlog})}: Conjunto de elementos seleccionados de la pila del producto para ser desarrollados durante el sprint, junto con un plan para entregar el incremento del producto y lograr el objetivo del Sprint.
	\item \textbf{Incremento}: La suma de todos los elementos de la pila del producto completados durante un sprint y todos los sprints anteriores, que cumple con los criterios de aceptación y asegura que el producto es potencialmente entregable.
\end{itemize}

\subsection{Eventos}
\begin{itemize}
	\item \textbf{\textit{Sprint}}: un periodo fijo durante el cual se crea un incremento del producto potencialmente entregable. Suele ser entre una y cuatro semanas.
	\item \textbf{Planificación del sprint}: sesion al inicio del sprint donde el equipo selecciona trabajo de la pila del producto para completar durante el sprint.
	\item \textbf{Reunión diaria}: breve reunion diaria para sincronizar actividades y crear un plan para el proximo dia, facilitando la colaboración.
	\item \textbf{Revisión del sprint}: al final del sprint, el equipo presenta el incremento a los interesados, recopilando retroalimentación para futuras iteraciones.
	\item \textbf{Retrospectiva del sprint}: oportunidad para el equipo scrum de inspeccionarse a sí mismo y crear un plan de mejoras para el proximo sprint.
\end{itemize}

\section{Herramientas}
Taiga, visual studio code, zappier, 