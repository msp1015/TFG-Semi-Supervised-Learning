\apendice{Documentación técnica de programación}

\section{Introducción}

\section{Estructura de directorios}

\section{Manual del programador}

\section{Compilación, instalación y ejecución del proyecto}

\section{Pruebas del sistema}
Esta sección del proyecto se centra en las pruebas de la aplicación web para garantizar la calidad y fiabilidad del sistema. Se ha optado por utilizar Selenium IDE debido a sus múltiples ventajas, siendo una heramienta poderosa y fácil de usar e instalar.

En el marco de este proyecto, se ha decidido heredar todos los casos de prueba definidos en el trabajo anterior, garantizando así la continuidad y la cobertura de pruebas ya establecidas. Y se han añadido nuevos casos de prueba para cubrir las nuevas funcionalidades implementadas en la versión 2.0.

En la figura \ref{fig:../img/anexos/testSelenium.png} se muestra la lista de pruebas vistas en Selenium IDE.

%\imagen{../img/anexos/testSelenium.png}{Conjunto de test organizados en Selenium IDE}

\begin{longtable}{>{\raggedright\arraybackslash}p{4cm} p{9.5cm}}
	\hline
	\rowcolor{gray!20}
	\textbf{ID} & CP-18\\
	\hline
	\rowcolor{white}
	\textbf{Prioridad} & Alta \\
	\hline
	\rowcolor{gray!20}
	\textbf{Fecha de Ejecución} & 03/07/2024 \\
	\hline
	\rowcolor{white}
	\textbf{Tester} & Mario Sanz Pérez \\
	\hline
	\rowcolor{gray!20}
	\textbf{Descripción} & x\\
	\hline
	\rowcolor{white}
	\textbf{Precondiciones} & \\
	\hline
	\rowcolor{gray!20}
	\textbf{Postcondiciones} & \\
	\hline
	\rowcolor{white}
	\textbf{Datos Utilizados} & \\
	\hline
	\rowcolor{gray!20}
	\textbf{Pasos} & \begin{enumerate}
		\item a
		\item b
		\item c
		\item d
	\end{enumerate}\\
	\hline
	\rowcolor{white}
	\textbf{Estado} & Exitoso\\
	\hline
\end{longtable}