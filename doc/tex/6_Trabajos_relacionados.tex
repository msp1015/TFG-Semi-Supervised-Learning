\capitulo{6}{Trabajos relacionados}
En esta sección, se analizarán los trabajos relacionados con el proyecto, destacando cómo cada uno de ellos ha influido y enriquecido el desarrollo. Es esencial reconocer y comprender estos trabajos, ya que ofrecen una base sobre la cual podemos construir, identifican oportunidades para la innovación y permiten evitar la repetición de errores pasados
\subsection{Visualizador de Algoritmos SemiSupervisados (VASS)}
Un pilar fundamental en el desarrollo de este proyecto es el trabajo realizado por David Martinez Acha. Este trabajo no solo ha sentado las bases conceptuales sino que también ha proporcionado una implementación práctica de gran valor. Se hereda directamente todo el esfuerzo de David, incluyendo sus investigaciones y los trabajos relacionados que identificó como relevantes en su estudio. Esta herencia es particularmente valiosa en lo que respecta a la implementación de la web, donde la infraestructura y el diseño preexistentes ofrecen una plataforma robusta desde la cual se puede avanzar sin partir de cero.

La implementación web desarrollada por David se destaca por su claridad, ofreciendo un punto de partida sólido para propias innovaciones. Aprovechar esta base preexistente permite enfocarse en la expansión y mejora de la funcionalidad, en lugar de invertir tiempo y recursos significativos en reconstruir lo que ya se ha logrado con éxito.

En resumen, la aplicación de David tiene 2 grandes funcionalidades separadas: los algoritmos y la gestión de usuarios. Lo que realmente importa para este proyecto es la implementación web de la parte de los algoritmos. Se tiene una pantalla principal donde permite seleccinar el algoritmo, una posterior donde se introduce el dataset, una ventana de configuración de parámetros del algoritmo y por úlitmo la visualización de los resultados.\\
La idea es mantener las mismas funcionalidades en el mismo orden, pudiendo modificar o extender ciertas partes gráficas que unifiquen o lo hagan más sencillo.

\subsection{ML Algorithm Visualizer}
Como ya se ha comentado en la sección anterior, esta página web fue importante en el planteamiento del primer diseño de la web. Se trata de una página web de visualización de algoritmos de aprendizaje automático, cuyo sitio web es \url{https://ml-visualizer.herokuapp.com/}. Con un diseño compacto permite seleccionar el modelo junto con la configuración de sus paramétros y unos segundos más tarde mostrará el resultado en la misma página, todo ello en una misma ventana como se puede ver en la imagen \ref{fig:../img/memoria/MLVisual-web.png}. Es aquí donde se diferencia del objetivo de este proyecto, donde se busca tener las ejecuciones separadas paso por paso.

\imagen{../img/memoria/MLVisual-web.png}{Captura de pantalla de la web \textit{ML Algorithm Visualizer}.}{1}

\subsection{Aprendizaje semisupervisado en ciberseguridad}
Este trabajo corresponde con el Trabajo Fin de Grado de Patricia Hernando. Pese a ser una herramienta destinada a la ciberseguridad, tiene relación con el actual trabajo en el uso de algoritmos semisupervisados. Permite ver la implementación y la explicación del código del Co-forest. Como ya se ha comentado, son de gran ayuda los estudios realizados para mejorar este algoritmo, ya que sirven para mostrar en la web una opción que permita seleccionar distintos valores y localizar diferencias en los resultados.