\apendice{Anexo de sostenibilización curricular}

\section{Introducción}
El proyecto realizado se centra en el desarrollo de una herramienta web destinada a la docencia de algoritmos de aprendizaje semisupervisados. A lo largo del desarrollo del proyecto, se han abordado varios aspectos relacionados con la sostenibilidad, tanto en el ámbito social como ambiental y económico.
El documento \url{https://www.crue.org/wp-content/uploads/2020/02/Directrices_Sosteniblidad_Crue2012.pdf} define las competencias en sostenibilidad que deben ser integradas en la formación universitaria, las cuales se comentan a continuación.

\subsection{Competencia en la Contextualización Crítica del Conocimiento}
Durante el desarrollo del proyecto, se ha realizado un esfuerzo significativo en comprender cómo la tecnología desarrollada puede influir en la educación y la sociedad. La herramienta no solo facilita el aprendizaje de algoritmos complejos, sino que también promueve el acceso equitativo a recursos educativos avanzados, lo cual es esencial para un desarrollo social justo y equitativo. Esta competencia se ha reflejado en la adaptación de la plataforma para ser accesible a usuarios con diferentes niveles de habilidades y conocimientos.

\subsection{Competencia en la Utilización Sostenible de Recursos}
El diseño y la implementación de la versión 2.0 del visualizador se ha llevado a cabo con un enfoque en la eficiencia de los recursos tecnológicos. Se ha utilizado un servidor que proporciona los servicios mínimos de la aplicación, sin exceder de recursos. Además, se ha priorizado el uso de software libre y herramientas de código abierto, reduciendo la dependencia de tecnologías propietarias y fomentando un entorno de desarrollo más sostenible.

\subsection{Competencia en la Participación en Procesos Comunitarios}
El proyecto fomenta la participación y colaboración dentro de la comunidad educativa. Se ha distribuido bajo una licencia MIT, lo que permite a otros educadores y desarrolladores utilizar y modificar el código fuente de la herramienta. Además, se ha promovido la retroalimentación y la contribución de la comunidad a través de la publicación del código en un repositorio público.
\subsection{Competencia en la Aplicación de Principios Éticos}
La implementación del visualizador de algoritmos se ha llevado a cabo siguiendo estrictos principios éticos. Se ha asegurado que los datos utilizados para el desarrollo y pruebas del sistema sean anónimos y se respete la privacidad de los usuarios.
