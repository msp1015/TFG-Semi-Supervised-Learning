\capitulo{2}{Objetivos del proyecto}

Este apartado explica de forma precisa y concisa cuales son los objetivos que se persiguen con la realización del proyecto. Se puede distinguir entre los objetivos marcados por los requisitos del software a construir y los objetivos de carácter técnico que plantea a la hora de llevar a la práctica el proyecto.

Para comprender los objetivos concretos es útil dar un contexto y contar el objetivo general del proyecto. Este trabajo se encuentra en el ámbito del aprendizaje automático, más concretamente en el aprendizaje semisupervisado, tratando de comprender su utilidad y profundizar en algunos de sus algoritmos; y en el ámbito del desarrollo web, con la intención de mejorar una página web ya existente que permite visualizar los algoritmos anteriores. Por lo tanto se podrían detallar tres objetivos generales:
\begin{itemize}
	\item Investigación exhaustiva sobre aprendizaje semisupervisado y sus algoritmos.
	\item Implementación de los algoritmos elegidos.
	\item Desarrollo web para la visualización del resultado de estos algoritmos.
\end{itemize}

\section{Objetivos técnicos}
Se definen los siguientes objetivos técnicos:

\begin{enumerate}
	\item Implementar distintos algoritmos de aprendizaje semisupervisado, basados en artículos científicos para conseguir una mayor eficiencia.
	\item Comparar las implementaciones propias con otras ajenas para comprobar su funcionamiento
	\item Continuar el desarrollo de la web ya implementada: para mejorar su funcionamiento y añadir nuevas funcionalidades.
	\item  Aprender a realizar experimentaciones de Aprendizaje Automático de forma rigurosa.
\end{enumerate}

\section{Objetivos de desarrollo \textit{software}}
Se definen los siguientes objetivos de desarrollo de software:

\begin{enumerate}
	\item Implementación de nuevos algoritmos utilizando librerías como \textit{Scickit-Learn} y la úlima versión de Python a fecha de inicio del proyecto (\textit{Python 3.11.5})
	\item Implementar un código limpio y estandarizado.
	\item Crear nuevas interfaces de usuario de visualización de algoritmos, basadas en la idea original.
	\item Mejorar ciertas funcionalidades de la web anterior.
	\item Conocer métodos de internacionalización, como \textit{babel}.
	\item Documentar el proceso de desarrollo: de manera resumida y con información útil.
	\item Familiarizarse con la metodología ágil \textit{Scrum}.
\end{enumerate}
