\capitulo{7}{Conclusiones y Líneas de trabajo futuras}

En este apartado, se presentan las conclusiones más relevantes del proyecto, destacando los logros y los aprendizajes teniendo en cuenta los objetivos establecidos al inicio. Además, se proporciona un análisis crítico que sugiere posibles mejoras y propone líneas de trabajo futuras para seguir avanzando en esta área.

\section{Conclusiones}

\section{Líneas de trabajo futuras}
Hacer una aplicación web siempre conlleva mejoras sea del estilo que sea, siempre se pueden añadir funcionalidades o mejorar las que existen. Por eso este proyecto no va a ser menos, estando abierto a muchas opciones de cambio. El hecho de que el proyecto sea para la docencia deja aún más puertas abiertas, ya que el mundo de la inteligencia artificial/aprendizaje automático evoluciona a un ritmo muy alto. De nuevo se vuelven a contar con las líneas de futuro presentadas en el trabajo anterior,~\cite{TFG:David}, y que no se han implementado en este.

\begin{itemize}
	\item Implementar nuevos algoritmos semisupervisados. Como se ha visto en la sección~\ref{sec3}, existe una amplia clasificación de este tipo de algoritmos. Hasta ahora se puede considerar que existen métodos de envoltura (\textit{wrapper}), \textit{ensembles} y métodos basados en grafos. Si esta gama de tipos se amplia se podría hacer una web bastante llamativa con una selección inicial diferente a la de ahora, contando con un nivel más alto.
	\item Permitir una carga de archivos más amplia. Por el momento, si el usuario sube un archivo para su clasificación, pero este contiene atributos categóricos, el proceso no sigue adelante, avisando del error. Para permitir mayor flexibilidad al usuario, se podría implementar un sistema de lectura de archivos mejorado que procese estos datos y los transforme en atributos numéricos.
	\item Visualizar comparaciones de resultados de ejecuciones. Aprovechando que en el trabajo anterior se trabajó en una gestión de usuarios los cuales pueden almacenar las ejecuciones llevadas a cabo, implementar una ventana en la que se compare las métricas de cada uno o cómo ha clasificado cada uno un dato, podría ser interesante para el usuario.
	\item Tutoriales en fase de configuración. Un estudiante puede estar interesado en algunos de los algoritmos en concreto, y aún con las explicaciones y el pseudocódigo, puede que en la configuración del algoritmo no se entienda bien que hace cada parámetro. Añadir un tutorial o una guía de uso de cada parámetro en cada configuración aclararía los conceptos para poder ejecutar adecuadamente.
	
	
\end{itemize}