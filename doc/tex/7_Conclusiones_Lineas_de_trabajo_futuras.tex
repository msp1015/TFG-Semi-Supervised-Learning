\capitulo{7}{Conclusiones y Líneas de trabajo futuras}

En este apartado, se presentan las conclusiones más relevantes del proyecto, destacando los logros y los aprendizajes teniendo en cuenta los objetivos establecidos al inicio. Además, se proporciona un análisis crítico que sugiere posibles mejoras y propone líneas de trabajo futuras para seguir avanzando en esta área.

\section{Conclusiones}

En esta sección se presentan las conclusiones técnicas y personales derivadas del desarrollo del proyecto.

\subsection{Conclusiones Técnicas}

A lo largo de este proyecto, se han abordado y cumplido varios objetivos técnicos clave, en los que se puede sacar varias conclusiones por cada uno:

\begin{itemize}
	\item \textbf{Implementación de Algoritmos de Aprendizaje Semisupervisado}: Se han implementado varios algoritmos basados en artículos científicos, logrando buenos resultados en la clasificación y predicción de datos. La comparación con implementaciones de terceros ha permitido validar la eficacia de los algoritmos desarrollados.
	\item \textbf{Desarrollo Continuo de la Web}: Se ha continuado el desarrollo de la web ya implementada, mejorando su funcionamiento y añadiendo nuevas funcionalidades. Esto incluye la optimización de la interfaz de usuario y la incorporación de nuevas características basadas en las necesidades del usuario final.
	\item \textbf{Experimentación en Aprendizaje Automático}: Se han aprendido y aplicado métodos rigurosos para la experimentación en aprendizaje automático, asegurando resultados reproducibles y confiables. Esto incluye la preparación de datos, la validación cruzada y el análisis de métricas de desempeño.
	\item \textbf{Conocimiento de herramientas y técnicas}: sobretodo en el desarrollo web, con el que se había <<jugado>> muy poco en la carrera, se ha logrado tener un buen conocimiento base de algunas herramientas comunes y de buenas prácticas en su uso.
	\item \textbf{Despliegue de Aplicaciones Web}: Se adquirieron conocimientos sobre el proceso completo de despliegue de una aplicación web, incluyendo la configuración de servidores, la implementación de servidores web y de aplicaciones, y la gestión de dominios y certificados de seguridad. 
\end{itemize}

\subsection{Conclusiones Personales}
En lo personal, siento que el trabajo realizado es bueno, pero que, siendo crítico, se podría haber hecho mejor en ciertos aspectos.

Hay que tener en cuenta que gran parte del trabajo en este tipo de proyectos parece <<invisible>>, como por ejemplo la lectura de artículos científicos o el hecho de depurar algoritmos, ya que difícilmente salen bien a la primera. Este tipo de tareas llevan un tiempo considerable que hay que tener en cuenta a la hora de evaluar el trabajo.

Siempre se escucha decir a profesorado y profesionales que cuando heredas un trabajo que no has desarrollado tú mismo, muchas veces es más difícil empezar a implementar, en este caso no ha sido del todo así. Aunque estuve un tiempo considerable para conseguir navegar fluidamente por el proyecto, el hecho de que David hiciera un código mantenible y extensible, ayudó mucho en las nuevas funcionalidades. Aún así, creo que no he terminado de aprovechar esta ventaja de tener un proyecto base bien estructurado (todo sea dicho que la decisión de hacer una segunda versión no surge desde el inicio). Se podrían haber implementado más algoritmos, ya sea de grafos o de \textit{ensembles} y también se podría haber desarrollado algún paso más en el proceso seguido en la web.

Respecto a la web, me llevo un gran aprendizaje con respecto a los lenguajes, herramientas y técnicas que se utilizan comúnmente. El no tener conocimientos previos me limitó al inicio para poder entender y avanzar el código existente, pero finalmente se ha conseguido una base sólida. Al contrario pasa con el ámbito del aprendizaje automático, que, aunque se ha tenido que leer e investigar, se partía con una buena base vista en asignaturas como Sistemas Inteligentes, Computación Neuronal y Minería de Datos.

En conclusión, aunque se ha logrado cumplir con los objetivos principales del proyecto, siempre hay margen para mejorar. La experiencia adquirida en el despliegue de aplicaciones web y la implementación de algoritmos de aprendizaje automático ha sido invaluable. Sin embargo, es evidente que se podría haber optimizado el tiempo y los recursos para incluir más funcionalidades y optimizar aún más el proyecto.

\section{Líneas de trabajo futuras}
Hacer una aplicación web siempre conlleva mejoras sea del estilo que sea, siempre se pueden añadir funcionalidades o mejorar las que existen. Por eso este proyecto no va a ser menos, estando abierto a muchas opciones de cambio. El hecho de que el proyecto sea para la docencia deja aún más puertas abiertas, ya que el mundo de la inteligencia artificial/aprendizaje automático evoluciona a un ritmo muy alto. De nuevo se vuelven a contar con las líneas de futuro presentadas en el trabajo anterior,~\cite{TFG:David}, y que no se han implementado en este.

\begin{itemize}
	\item \textbf{Implementar nuevos algoritmos semisupervisados}. Como se ha visto en la sección~\hyperref[sec3:tax]{3.2.1}, existe una amplia clasificación de este tipo de algoritmos. Hasta ahora se puede considerar que existen métodos de envoltura (\textit{wrapper}), \textit{ensembles} y métodos basados en grafos. Si esta gama de tipos se amplía se podría hacer una web bastante llamativa con una selección inicial diferente a la de ahora, contando con un nivel más alto.
	\item \textbf{Permitir una carga de archivos más amplia}. Por el momento, si el usuario sube un archivo para su clasificación, pero este contiene atributos categóricos, el proceso no sigue adelante, avisando del error. Para permitir mayor flexibilidad al usuario, se podría implementar un sistema de lectura de archivos mejorado que procese estos datos y los transforme en atributos numéricos.
	\item \textbf{Visualizar comparaciones de resultados de ejecuciones}. Aprovechando que en el trabajo anterior se trabajó en una gestión de usuarios los cuales pueden almacenar las ejecuciones llevadas a cabo, implementar una ventana en la que se compare las métricas de cada uno o cómo ha clasificado cada uno un dato, podría ser interesante para el usuario.
	\item \textbf{Tutoriales en fase de configuración}. Un estudiante puede estar interesado en algunos de los algoritmos en concreto, y aún con las explicaciones y el pseudocódigo, puede que en la configuración del algoritmo no se entienda bien que hace cada parámetro. Añadir un tutorial o una guía de uso de cada parámetro en cada configuración aclararía los conceptos para poder ejecutar adecuadamente.
	\item \textbf{Seguimiento de los resultados}: Actualmente la aplicación cumple su objetivo de poder enseñar como funciona un algoritmo de aprendizaje semisupervisado, pero ya que realmente se están prediciendo y evaluando datos de entrada reales, se puede realizar un paso más en el que se vean los resultados finales de cada entrada de datos, viendo su valor real y su valor predicho. Además de poder asociarlos con las representaciones para tener situados a todos los datos.
	\item \textbf{Ciberseguridad en la web}: Hay muchos peligros a los que las páginas webs se exponen, sobretodo aquellas que tratan con datos sensibles, por ello estaría bien implementar más medidas de seguridad además de CSRF. Realmente esta línea de trabajo está pensada en el sentido de aprender las técnicas que existen de \textit{hacking} web y cómo evitarlas, ya que no es un objetivo real de ciberdelincuentes.
	\item \textbf{Detallar pasos de los grafos}: la visualización de los grafos resultó compleja y llevó gran parte del tiempo, pero estos algoritmos dan pie a detallar mucho más lo que está pasando por detrás del algoritmo. Por ejemplo, se podría indicar en todo momento cual es la lista de enlaces de un nodo seleccionado, junto con su distancia calculada, para entender por qué se une a unos nodos y a otros no.
\end{itemize}