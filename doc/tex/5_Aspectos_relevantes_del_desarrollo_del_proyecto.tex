\capitulo{5}{Aspectos relevantes del desarrollo del proyecto}
En la sección que sigue, se abordan los aspectos más significativos que han marcado el desarrollo de este proyecto. Se detalla cómo cada elección ha influido en la trayectoria y los resultados del proyecto.

\section{Trabajo Preexistente}
La elección de este trabajo se realiza por el gran interés en la inteligencia artificial y el aprendizaje automático, pero también por el hecho de que ya existía un trabajo realizado por otro alumno un año atrás (David Martínez Acha -- \url{https://vass.dmacha.dev/}). Esto ayudaría mucho en el desarrollo ya que serviría como referencia para muchas dudas.\\ 
El plan original era hacer mi propia página web educativa desde cero, pero mostrando otra serie de algoritmos (ensembles y grafos) en lugar de los ya existentes. Con el desarrollo del primer algoritmo surge la idea por parte del tutor de basar el proyecto en esta otra página desarrollada por David Martinez. De esta manera, el tiempo que hubiera empleado en aprender e implementar la web desde cero, se emplea en comprender todo el código hecho por David, aprovechando también la gestión de cuentas de usuarios. Aún así, se deja total libertad para cambiar e implementar lo que haga falta para mejorar el proyecto original.\\
El tiempo empleado en ajustarse al nuevo código fue de dos sprints, ya que no solo trataba de leer código, sino de comprender las técnicas de HTML, css y javascript que se utilizan, junto con bibliotecas como \textit{Bootstrap}.
Aún así, el tiempo ganado es considerable y da pie a poder implementar más algoritmos y pensar en ideas que mejoran la web.
Inicialmente se piensa que con un fork a su repositorio de GitHub,~\cite{GH:VASS}, se puede trabajar mejor, pero de esta manera se perderían las tareas y commits hechos hasta la fecha en el repositorio de este proyecto. Por esto se decide descargar el contenido y copiarlo a el proyecto ya en desarrollo.\\
La documentación de David, \cite{TFG:David}, sirve de gran ayuda y también se heredan partes de ella, como los trabajos relacionados y conceptos teóricos. También se tiene en cuenta las líneas de trabajo futuras desarrolladas para poder implementarlas en este trabajo.\\
La primera idea de cambio que surge es la de modificar la funcionalidad de configuración de un algoritmo. Visualizando páginas web, se da con una página que muestra algoritmos de aprendizaje automático~ \cite{web:ml-visualizer}, pero no de la misma manera que David. En esta página se permite configurar y ver resultados y estadísticas en una única ventana, con la gran diferencia de que no muetra el paso a paso en cada algoritmo, sino directamente el resultado final de la clasificación.
El primer prototipo se hace pensando en esta idea,quedando algo parecido a:
\imagen{../img/anexos/prototipo_coforest.png}{Prototipo de visualizacion de algoritmo Co-Forest}{1}
Posteriormente, gracias al tutor nos damos cuenta de que si la idea es mostrar el paso a paso, que los parametros de configuración estén en la misma pantalla, no es de gran ayuda. Por esto, se decide volver a la versión del trabajo base y modificar las plantillas necesarias para adaptarlas a los nuevos algoritmos.

\section{Algoritmos}
En esta sección se comentarán los aspectos más relevantes en la implementación de los algoritmos semisupervisados.
\subsection{Co-Forest}
En la implementación de este algoritmo de nuevo se parte con una gran ventaja. El año anterior otra alumna había implementado el mismo algoritmo para otro proyecto. Dentro de esta aplicación, Patricia Hernando, hizo sus propios estudios, los cuales se han aprovechado en este trabajo.
Basado fuertemente en el pseudocodigo del artículo \cite{IEEE:CoForest}, la implementación se ve alterada por el parámetro W, el cual establece la confianza en las muestras para ser seleccionada o no. Resumiendo el estudio de Patricia, como se puede ver en el pseudocodigo del articulo, hay una ecuación donde el valor de W esta dividiendo. Esto es un problema ya que según establece el algoritmo puede llegar a valer cero, provocando así una indeterminación. Uno de los estudios de Patricia determina que una de las mejores soluciones a esto es iniciar el parámetro W al minimo entre 100 y el 10\% de la cantidad de muestras etiquetadas que hay. Como se puede ver en el pseudocodigo de la sección tres, esto se aprovecha, evitando así posibles problemas.
Para determinar si el algoritmo definitivo es bueno, se compara con el de Patricia, evaluando como varía el valor de \textit{accuracy} en cada iteración del algoritmo. Los resultados se muestran en las siguientes gráficas:
\imagenflotante{../img/memoria/ComparacionCoForest.png}{Comparación algoritmo CoForest utilizando diversos datasets}{1}
