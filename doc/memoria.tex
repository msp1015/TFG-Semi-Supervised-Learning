\documentclass[a4paper,12pt,twoside]{memoir}

% Castellano
\usepackage[spanish,es-tabla]{babel}
\selectlanguage{spanish}
\usepackage[utf8]{inputenc}
\usepackage[T1]{fontenc}
\usepackage{lmodern} % Scalable font
\usepackage{microtype}
\usepackage{placeins}
\usepackage{mdframed}

\RequirePackage{booktabs}
\RequirePackage[table]{xcolor}
\RequirePackage{xtab}
\RequirePackage{multirow}

% Links
\PassOptionsToPackage{hyphens}{url}\usepackage[colorlinks]{hyperref}
\hypersetup{
	allcolors = {red}
}

% Pseudocódigo
\usepackage[longend, noline, ruled, linesnumbered, resetcount, spanish, spanishkw]{algorithm2e}

% Ecuaciones
\usepackage{amsmath}

% Rutas de fichero / paquete
\newcommand{\ruta}[1]{{\sffamily #1}}

% Párrafos
\nonzeroparskip

% Huérfanas y viudas
\widowpenalty100000
\clubpenalty100000

% Imágenes

% Comando para insertar una imagen en un lugar concreto.
% Los parámetros son:
% 1 --> Ruta absoluta/relativa de la figura
% 2 --> Texto a pie de figura
% 3 --> Tamaño en tanto por uno relativo al ancho de página
% 4 --> Texto en el índice
\usepackage{graphicx}
\newcommand{\imagen}[4]{
	\begin{figure}[!h]
		\centering
		\includegraphics[width=#3\textwidth]{#1}
		\caption[#4]{#2}\label{fig:#1}
	\end{figure}
	\FloatBarrier
}

% Comando para insertar una imagen sin posición.
% Los parámetros son:
% 1 --> Ruta absoluta/relativa de la figura
% 2 --> Texto a pie de figura
% 3 --> Tamaño en tanto por uno relativo al ancho de página
\newcommand{\imagenflotante}[3]{
	\begin{figure}
		\centering
		\includegraphics[width=#3\textwidth]{#1}
		\caption{#2}\label{fig:#1}
	\end{figure}
}

% El comando \figura nos permite insertar figuras comodamente, y utilizando
% siempre el mismo formato. Los parametros son:
% 1 --> Porcentaje del ancho de página que ocupará la figura (de 0 a 1)
% 2 --> Fichero de la imagen
% 3 --> Texto a pie de imagen
% 4 --> Etiqueta (label) para referencias
% 5 --> Opciones que queramos pasarle al \includegraphics
% 6 --> Opciones de posicionamiento a pasarle a \begin{figure}
\newcommand{\figuraConPosicion}[6]{%
  \setlength{\anchoFloat}{#1\textwidth}%
  \addtolength{\anchoFloat}{-4\fboxsep}%
  \setlength{\anchoFigura}{\anchoFloat}%
  \begin{figure}[#6]
    \begin{center}%
      \Ovalbox{%
        \begin{minipage}{\anchoFloat}%
          \begin{center}%
            \includegraphics[width=\anchoFigura,#5]{#2}%
            \caption{#3}%
            \label{#4}%
          \end{center}%
        \end{minipage}
      }%
    \end{center}%
  \end{figure}%
}

%
% Comando para incluir imágenes en formato apaisado (sin marco).
\newcommand{\figuraApaisadaSinMarco}[5]{%
  \begin{figure}%
    \begin{center}%
    \includegraphics[angle=90,height=#1\textheight,#5]{#2}%
    \caption{#3}%
    \label{#4}%
    \end{center}%
  \end{figure}%
}
% Para las tablas
\newcommand{\otoprule}{\midrule [\heavyrulewidth]}
%
% Nuevo comando para tablas pequeñas (menos de una página).
\newcommand{\tablaSmall}[5]{%
 \begin{table}
  \begin{center}
   \rowcolors {2}{gray!35}{}
   \begin{tabular}{#2}
    \toprule
    #4
    \otoprule
    #5
    \bottomrule
   \end{tabular}
   \caption{#1}
   \label{tabla:#3}
  \end{center}
 \end{table}
}

%
% Nuevo comando para tablas pequeñas (menos de una página).
\newcommand{\tablaSmallSinColores}[5]{%
 \begin{table}[H]
  \begin{center}
   \begin{tabular}{#2}
    \toprule
    #4
    \otoprule
    #5
    \bottomrule
   \end{tabular}
   \caption{#1}
   \label{tabla:#3}
  \end{center}
 \end{table}
}

\newcommand{\tablaApaisadaSmall}[5]{%
\begin{landscape}
  \begin{table}
   \begin{center}
    \rowcolors {2}{gray!35}{}
    \begin{tabular}{#2}
     \toprule
     #4
     \otoprule
     #5
     \bottomrule
    \end{tabular}
    \caption{#1}
    \label{tabla:#3}
   \end{center}
  \end{table}
\end{landscape}
}

%
% Nuevo comando para tablas grandes con cabecera y filas alternas coloreadas en gris.
\newcommand{\tabla}[6]{%
  \begin{center}
    \tablefirsthead{
      \toprule
      #5
      \otoprule
    }
    \tablehead{
      \multicolumn{#3}{l}{\small\sl continúa desde la página anterior}\\
      \toprule
      #5
      \otoprule
    }
    \tabletail{
      \hline
      \multicolumn{#3}{r}{\small\sl continúa en la página siguiente}\\
    }
    \tablelasttail{
      \hline
    }
    \bottomcaption{#1}
    \rowcolors {2}{gray!35}{}
    \begin{xtabular}{#2}
      #6
      \bottomrule
    \end{xtabular}
    \label{tabla:#4}
  \end{center}
}

%
% Nuevo comando para tablas grandes con cabecera.
\newcommand{\tablaSinColores}[6]{%
  \begin{center}
    \tablefirsthead{
      \toprule
      #5
      \otoprule
    }
    \tablehead{
      \multicolumn{#3}{l}{\small\sl continúa desde la página anterior}\\
      \toprule
      #5
      \otoprule
    }
    \tabletail{
      \hline
      \multicolumn{#3}{r}{\small\sl continúa en la página siguiente}\\
    }
    \tablelasttail{
      \hline
    }
    \bottomcaption{#1}
    \begin{xtabular}{#2}
      #6
      \bottomrule
    \end{xtabular}
    \label{tabla:#4}
  \end{center}
}

%
% Nuevo comando para tablas grandes sin cabecera.
\newcommand{\tablaSinCabecera}[5]{%
  \begin{center}
    \tablefirsthead{
      \toprule
    }
    \tablehead{
      \multicolumn{#3}{l}{\small\sl continúa desde la página anterior}\\
      \hline
    }
    \tabletail{
      \hline
      \multicolumn{#3}{r}{\small\sl continúa en la página siguiente}\\
    }
    \tablelasttail{
      \hline
    }
    \bottomcaption{#1}
  \begin{xtabular}{#2}
    #5
   \bottomrule
  \end{xtabular}
  \label{tabla:#4}
  \end{center}
}



\definecolor{cgoLight}{HTML}{EEEEEE}
\definecolor{cgoExtralight}{HTML}{FFFFFF}

%
% Nuevo comando para tablas grandes sin cabecera.
\newcommand{\tablaSinCabeceraConBandas}[5]{%
  \begin{center}
    \tablefirsthead{
      \toprule
    }
    \tablehead{
      \multicolumn{#3}{l}{\small\sl continúa desde la página anterior}\\
      \hline
    }
    \tabletail{
      \hline
      \multicolumn{#3}{r}{\small\sl continúa en la página siguiente}\\
    }
    \tablelasttail{
      \hline
    }
    \bottomcaption{#1}
    \rowcolors[]{1}{cgoExtralight}{cgoLight}

  \begin{xtabular}{#2}
    #5
   \bottomrule
  \end{xtabular}
  \label{tabla:#4}
  \end{center}
}



\graphicspath{ {./img/} }

% Capítulos
\chapterstyle{bianchi}
\newcommand{\capitulo}[2]{
	\setcounter{chapter}{#1}
	\setcounter{section}{0}
	\setcounter{figure}{0}
	\setcounter{table}{0}
	\chapter*{\thechapter.\enskip #2}
	\addcontentsline{toc}{chapter}{\thechapter.\enskip #2}
	\markboth{#2}{#2}
}

% Apéndices
\renewcommand{\appendixname}{Apéndice}
\renewcommand*\cftappendixname{\appendixname}

\newcommand{\apendice}[1]{
	%\renewcommand{\thechapter}{A}
	\chapter{#1}
}

\renewcommand*\cftappendixname{\appendixname\ }

% Formato de portada
\makeatletter
\usepackage{xcolor}
\newcommand{\tutor}[1]{\def\@tutor{#1}}
\newcommand{\course}[1]{\def\@course{#1}}
\definecolor{cpardoBox}{HTML}{E6E6FF}
\def\maketitle{
  \null
  \thispagestyle{empty}
  % Cabecera ----------------
\noindent\includegraphics[width=\textwidth]{cabecera}\vspace{1cm}%
  \vfill
  % Título proyecto y escudo informática ----------------
  \colorbox{cpardoBox}{%
    \begin{minipage}{.8\textwidth}
      \vspace{.5cm}\Large
      \begin{center}
      \textbf{TFG del Grado en Ingeniería Informática}\vspace{.6cm}\\
      \textbf{\LARGE\@title{}}
      \end{center}
      \vspace{.2cm}
    \end{minipage}

  }%
  \hfill\begin{minipage}{.20\textwidth}
    \includegraphics[width=\textwidth]{escudoInfor}
  \end{minipage}
  \vfill
  % Datos de alumno, curso y tutores ------------------
  \begin{center}%
  {%
    \noindent\LARGE
    Presentado por \@author{}\\ 
    en Universidad de Burgos --- \@date{}\\
    Tutor: \@tutor{}\\
  }%
  \end{center}%
  \null
  \cleardoublepage
  }
\makeatother

\newcommand{\nombre}{Mario Sanz Pérez} %%% cambio de comando
\newcommand{\Tutor}{Dr. Álvar Arnaiz González}

% Datos de portada
\title{Ampliación web para la docencia de métodos de aprendizaje semisupervisado}
\author{\nombre}
\tutor{\Tutor}
\date{\today}

\begin{document}

\maketitle


\newpage\null\thispagestyle{empty}\newpage


%%%%%%%%%%%%%%%%%%%%%%%%%%%%%%%%%%%%%%%%%%%%%%%%%%%%%%%%%%%%%%%%%%%%%%%%%%%%%%%%%%%%%%%%
\thispagestyle{empty}


\noindent\includegraphics[width=\textwidth]{cabecera}\vspace{1cm}

\noindent El \Tutor, profesor del departamento de Ingeniería Informática, área de Lenguajes y
Sistemas informáticos.

\noindent Expone:

\noindent Que el alumno D. \nombre, con DNI 71482918E, ha realizado el Trabajo final de Grado en Ingeniería Informática titulado <<Ampliación web para la docencia de métodos de aprendizaje semisupervisado>>. 

\noindent Y que dicho trabajo ha sido realizado por el alumno bajo la dirección del que suscribe, en virtud de lo cual se autoriza su presentación y defensa.

\begin{center} %\large
En Burgos, {\large \today}
\end{center}

\vfill\vfill\vfill

% Author and supervisor
\begin{center}
\begin{minipage}{0.45\textwidth}
\begin{flushleft} %\large
Vº. Bº. del Tutor:\\[2cm]
\Tutor
\end{flushleft}
\end{minipage}
\hfill
\end{center}


\vfill

% para casos con solo un tutor comentar lo anterior
% y descomentar lo siguiente
%Vº. Bº. del Tutor:\\[2cm]
%D. nombre tutor


\newpage\null\thispagestyle{empty}\newpage




\frontmatter

% Abstract en castellano
\renewcommand*\abstractname{Resumen}
\begin{abstract}
En la práctica, el aprendizaje semisupervisado es crucial debido a los retos y altos costos que implica obtener datos etiquetados de calidad. A pesar de su relevancia, a estos algoritmos no se les ha prestado suficiente atención, tanto en la investigación académica como en la educación, donde el énfasis generalmente recae en el aprendizaje supervisado y no supervisado.

En un proyecto anterior, se desarrolló una biblioteca con cuatro de los más populares algoritmos de aprendizaje semisupervisados: Self-Training, Co-Training, Democratic Co-Learning y Tri-Training, además de una aplicación web para visualizar el proceso de entrenamiento de los mismos.

En este trabajo, se ha continuado la herramienta docente, añadiendo nuevos algoritmos para ampliar su alcance y funcionalidad. Se ha incorporado el algoritmo Co-Forest para métodos de \textit{ensembles} y, para métodos transductivos basados en grafos, se han implementado los algoritmos GBILI y RGCLI en la fase de construcción de grafos, y el algoritmo LGC (\textit{Local and Global Consistency}) en la fase de inferencia de etiquetas.

El objetivo principal sigue siendo el mismo: mostrar cómo van cambiando los datos no etiquetados en la fase de entrenamiento, facilitando su comprensión. La visualización de las diferentes fases de estos algoritmos se han realizado en la misma aplicación web utilizada anteriormente, aunque con modificaciones para adaptarse a los nuevos algoritmos incorporados.

Se parte de la aplicación web \url{https://vass.dmacha.dev} y se evoluciona a una segunda versión disponible en \url{https://vass2.dev}.
\end{abstract}

\renewcommand*\abstractname{Descriptores}
\begin{abstract}
aprendizaje automático, aprendizaje semisupervisado, ensembles, grafos, web
\end{abstract}

\clearpage

% Abstract en inglés
\renewcommand*\abstractname{Abstract}
\begin{abstract}
In practice, semi-supervised learning is crucial due to the challenges and high costs involved in obtaining quality labeled data. Despite their relevance, these algorithms have not been given enough attention, both in academic research and in education, where the emphasis generally falls on supervised and unsupervised learning.

In a previous project, a library was developed with four of the most popular semi-supervised learning algorithms: Self-Training, Co-Training, Democratic Co-Learning and Tri-Training, as well as a web application to visualize their training process.

In this work, the educational tool has been continued, adding new algorithms to expand its scope and functionality. The Co-Forest algorithm for ensemble methods has been incorporated, and for graph-based transductive methods, the GBILI and RGCLI algorithms have been implemented in the graph construction phase, and the LGC (Local and Global Consistency) algorithm in the label inference phase.

The main objective remains the same: to show how the unlabeled data changes during the training phase, facilitating its understanding. The visualization of the different phases of these algorithms has been carried out in the same web application used previously, with modifications to accommodate the newly incorporated algorithms.

The web application \url{https://vass.dmacha.dev} has been used as a starting point and has evolved into a second version available at \url{https://vass2.dev}.
\end{abstract}

\renewcommand*\abstractname{Keywords}
\begin{abstract}
	machine learning, semi-supervised learning, ensembles, graphs, web
\end{abstract}


\clearpage

% Indices
\tableofcontents

\clearpage

\listoffigures

\clearpage

\listoftables
\clearpage

\mainmatter
\capitulo{1}{Introducción}

En los últimos años, el aprendizaje automático ha avanzado significativamente, destacando su capacidad para resolver problemas complejos en una variedad de dominios. Sin embargo, uno de los desafíos más persistentes es la obtención de datos etiquetados de alta calidad, los cuales son esenciales para el entrenamiento de modelos supervisados . En este contexto, el aprendizaje semisupervisado se presenta como una solución viable, al aprovechar tanto los datos etiquetados como los no etiquetados para mejorar la precisión de los modelos.

El aprendizaje semi-supervisado incluye una variedad de algoritmos que permiten inferir etiquetas para datos no etiquetados, basándose en la información disponible de los datos etiquetados. Entre estos algoritmos se encuentran los métodos de \textit{ensembles} y los métodos basados en grafos, cada uno con sus propias ventajas y aplicaciones específicas. Los métodos de \textit{ensembles} combinan múltiples modelos de aprendizaje para mejorar el rendimiento general. La idea principal es que, al combinar las predicciones de varios modelos, se puede reducir la varianza y el sesgo, logrando una mejor precisión y robustez. Por otro lado, los métodos basados en grafos representan los datos como nodos en un grafo, donde los enlaces (aristas) entre los nodos indican relaciones o similitudes entre los datos. Este enfoque es especialmente poderoso para capturar la estructura intrínseca de los datos.

Este proyecto se centra en la continuación y ampliación de una herramienta docente previamente desarrollada, la cual incluye una biblioteca de algoritmos semi-supervisados y una aplicación web para la visualización del proceso de entrenamiento. En la versión anterior, se implementaron cuatro algoritmos populares: \textit{Self-Training}, \textit{Co-Training}, \textit{Democratic Co-Learning} y \textit{Tri-Training}.

En esta nueva fase del proyecto, se han añadido nuevos algoritmos para ampliar la funcionalidad de la herramienta. Específicamente, se ha incorporado el algoritmo \textit{Co-Forest} para métodos de ensembles y, para métodos transductivos basados en grafos, se han implementado los algoritmos \textit{GBILI} y \textit{RGCLI} en la fase de construcción de grafos, así como el algoritmo \textit{LGC} (\textit{Local and Global Consistency}) en la fase de inferencia de etiquetas.

La estructura de este documento se organiza de la siguiente manera:

\begin{itemize}
	\item \textbf{Memoria}: documento principal del proyecto que se divide en:
	\begin{enumerate}
		\item \textbf{Introducción}: se ofrece una visión general del proyecto junto con la estructura de la documentación.
		\item \textbf{Objetivos}: se enumeran los objetivos generales, técnicos y de desarrollo \textit{software}.
		\item \textbf{Conceptos teóricos}: se explica en detalle los conceptos teóricos claves para comprender el desarrollo del proyecto (tanto en la documentación como en la aplicación final).
		\item \textbf{Técnicas y herramientas}: se describe las tecnologías y herramientas de desarrollo empleadas en el proyecto.
		\item \textbf{Aspectos relevantes del proyecto}: se discuten los aspectos más importantes del desarrollo del proyecto, como los estudios realizados o las dificultades encontradas en pleno desarrollo.
		\item \textbf{Trabajos relacionados}: Se revisan otros trabajos y proyectos relevantes en el mismo ámbito que puedan servir de ayuda en el desarrollo.
		\item \textbf{Conclusiones y líneas de trabajo futuras}: se evalua el cumplimiento de los objetivos propuestos y se sugieren posibles mejoras para el futuro.
	\end{enumerate}
	\item \textbf{Anexos}: documento que contiene contenido adicional del desarrollo del proyecto, dividido en:
	\begin{enumerate}
		\item \textbf{Plan de proyecto \textit{software}}: se detalla el plan de desarrollo del \textit{software}, incluyendo recursos y estrategias de gestión del proyecto.
		\item \textbf{Especificación de requisitos}: se enumeran y describen los requisitos que el sistema debe cumplir usando diagramas y tablas.
		\item \textbf{Especificación de diseño}: se presenta el diseño detallado de la arquitectura del sistema, describiendo los componentes principales y su interacción.
		\item \textbf{Documentación técnica de programación}: se proporciona documentación técnica detallada, con el objetivo de simplificar la integración de nuevos desarrolladores en el proyecto y acelerar su comprensión del mismo.
		\item \textbf{Documentación de usuario}: se ofrece una guía detallada para los usuarios de la aplicación, explicando cómo utilizar las diferentes funcionalidades.
	\end{enumerate}
\end{itemize}


Como recursos adicionales se incluyen:
\begin{itemize}
	\item \textbf{Repositorio del proyecto}: \url{https://github.com/msp1015/TFG-Semi-Supervised-Learning}
	\item \textbf{Web desplegada}: 
	\begin{itemize}
		\item \textbf{Versión 1.0}: \url{https://vass.dmacha.dev/}
		\item \textbf{Versión 2.0}: \url{URL}
	\end{itemize}
\end{itemize}
\capitulo{2}{Objetivos del proyecto}

Este apartado explica de forma precisa y concisa cuales son los objetivos que se persiguen con la realización del proyecto. Se puede distinguir entre los objetivos marcados por los requisitos del software a construir y los objetivos de carácter técnico que plantea a la hora de llevar a la práctica el proyecto.

Para comprender los objetivos concretos es útil dar un contexto y contar el objetivo general del proyecto. Este trabajo se encuentra en el ámbito del aprendizaje automático, más concretamente en el aprendizaje semisupervisado, tratando de comprender su utilidad y profundizar en algunos de sus algoritmos; y en el ámbito del desarrollo web, con la intención de mejorar una página web ya existente que permite visualizar los algoritmos anteriores. Por lo tanto se podrían detallar tres objetivos generales:
\begin{itemize}
	\item Investigación exhaustiva sobre aprendizaje semisupervisado y sus algoritmos.
	\item Implementación de los algoritmos elegidos.
	\item Desarrollo web para la visualización del resultado de estos algoritmos.
\end{itemize}

\section{Objetivos técnicos}
Se definen los siguientes objetivos técnicos:

\begin{enumerate}
	\item Implementar distintos algoritmos de aprendizaje semisupervisado, basados en artículos científicos para conseguir una mayor eficiencia.
	\item Comparar las implementaciones propias con otras ajenas para comprobar su funcionamiento
	\item Continuar el desarrollo de la web ya implementada: para mejorar su funcionamiento y añadir nuevas funcionalidades.
	\item Aprender a realizar experimentaciones de Aprendizaje Automático de forma rigurosa.
	\item Aprender y aplicar el proceso de despliegue de una aplicación web, incluyendo la configuración de un servidor, la implementación de servidores web y de aplicaciones, y la gestión de dominios y certificados de seguridad.
\end{enumerate}

\section{Objetivos de desarrollo \textit{software}}
Se definen los siguientes objetivos de desarrollo de software:

\begin{enumerate}
	\item Implementación de nuevos algoritmos utilizando librerías como \textit{Scickit-Learn} y la úlima versión de Python a fecha de inicio del proyecto (\textit{Python 3.11.5})
	\item Implementar un código limpio y estandarizado.
	\item Crear nuevas interfaces de usuario de visualización de algoritmos, basadas en la idea original.
	\item Mejorar ciertas funcionalidades de la web anterior.
	\item Conocer métodos de internacionalización, como \textit{babel}.
	\item Documentar el proceso de desarrollo: de manera resumida y con información útil.
	\item Familiarizarse con la metodología ágil \textit{Scrum}.
	\item Descubrir nuevas técnicas y herramientas en el proceso de desarrollo (\textit{frontend} y \textit{backend}).
\end{enumerate}

\capitulo{3}{Conceptos teóricos}

En esta sección se resumirán los conceptos teóricos básicos y necesarios para comprender el trabajo. Principalmente se hablará de aprendizaje automático y luego se profundizará en el aprendizaje semi-supervisado.

\section{Aprendizaje automático}
El aprendizaje automático (\textit{Machine Learning} en inglés) es el campo de la inteligencia artificial (IA) que se centra en el uso de datos y en el desarrollo de algoritmos para imitar la manera de aprender de los humanos \cite{ML:ibm}. La esencia radica en la capacidad de los sistemas informáticos para aprender de datos y realizar tareas sin intervención humana directa, si no descubriendo patrones y tendencias en los mismos. A estos sistemas se les conoce como \textbf{modelos}, los cuales pueden mejorar su rendimiento y adaptarse a nuevas situaciones basándose en la experiencia pasada.

Según \cite{ML:DataScientest}, existen cuatro etapas principales en el desarrollo de un modelo. El primer paso consiste en seleccionar y preparar el conjunto de datos (\textit{dataset}) que utilizará el modelo para aprender a resolver el problema para el que se ha diseñado. En el segundo paso se selecciona el algoritmo para ejecutar sobre el \textit{dataset}. Este dependerá del tamaño y el tipo de los datos de entrada y del tipo de problema que se está resolviendo. El tercer paso consiste en entrenar el algoritmo hasta que la mayoría de los resultados sean los esperados. El cuarto y último paso trata de usar el modelo sobre nuevos datos y hacer una evaluación para una posible mejora.

Según los datos que se seleccionen en el primer paso, podemos tener dos ramas distintas en el aprendizaje automático \cite{ML:SisInt}:
\begin{itemize}
	\item \textbf{Predictiva}: también caracterizada por utilizar el aprendizaje supervisado, es decir, datos de entrada etiquetados.
	\item \textbf{Descriptiva}: al contrario, utiliza el aprendizaje no supervisado, con datos de entrada no etiquetados.
\end{itemize}

 
\subsection{Aprendizaje supervisado}
El aprendizaje supervisado es un tipo de aprendizaje automático en el que los modelos son entrenados utilizando conjuntos de datos etiquetados, en los que se basarán las decisiones y predicciones. Los conjuntos de datos contienen ejemplos emparejados de variables de entrada (o características) y de salida (o etiquetas). La esencia de este tipo de aprendizaje se basa en la capacidad del modelo para aprender la relación funcional entre las entradas y las salidas, permitiéndole hacer predicciones precisas sobre nuevos datos no vistos \cite{SL:guide}. De ahí su clasificación como \guillemetleft predictiva\guillemetright ~en la sección anterior.
Se puede clasificar este tipo de aprendizaje en dos tipos:
\begin{itemize}
	\item \textbf{Clasificación}: los modelos asignan categorías o clases a las entradas no etiquetadas. Dentro de este tipo se puede encontrar la clasificación binaria y la multi-clase. La primera se ve en un caso como la clasificación de correos electrónicos marcadas como spam o no spam (solo una etiqueta). Y la segunda se puede ver en cualquier ejemplo en el que haya mas de dos clases, como al establecer si un paciente tiene alto, medio o bajo riesgo de muerte ante una operación.
	\item \textbf{Regresión}: es similar a la clasificación, pero en vez de asignar un valor discreto, ahora es un valor continuo. Un ámbito común en el que se suele dar es en la economía, con la predicción de acciones o ventas.
\end{itemize}

También es importante comentar las principales fases que forman este aprendizaje y los posibles problemas o desafíos que pueden surgir, ya que pueden servir para tener en cuenta en los algoritmos concretos a implementar.
En la mayoría de algoritmos que utilizan datos etiquetados, estos se dividen en tres conjuntos: entrenamiento, validación y prueba. El conjunto de entrenamiento se utiliza para ajustar los parámetros del modelo, el conjunto de validación para ajustar los hiperparámetros y prevenir el sobreajuste y el conjunto de prueba apra evaluar el rendimiento final.
El sobreajuste o \textit{\textbf{overfitting}} es uno de los principales problemas del aprendizaje automático y ocurre cuando el modelo se ajusta demasiado a los datos de entrenamiento, es decir, los memoriza en vez de generalizar.

\subsection{Aprendizaje no supervisado}
Para explicar este apredizaje se usará el artículo \cite{USL:guide}. El aprendizaje no supervisado hace referencia a los tipos de problemas en los que se utiliza un modelo para caracterizar o extraer relaciones en los datos.
A diferencia del aprendizaje supervisado, estos algoritmos descubren la estructura implícita de un conjunto de datos utilizando únicamente características de entrada y no clases o categorías. 
Ya que no existen etiquetas en los datos, los métodos no supervisados se utilizan normalmente para crear una representación concisa de los datos, posibilitando la generación de contenido creativo a partir de ellos. Por ejemplo, si tenemos una gran cantidad de fotografías sin clasificar, un modelo no supervisado encontraría relaciones entre las caracteristicas para poder organizar automáticamente las imágenes en grupos.
Se pueden clasificar en tres diferentes categorías:
\begin{itemize}
	\item \textbf{Clustering}: segmentación o agrupamiento. Consiste en la identificación de grupos o \textit{clusters} en función de sus similitudes y diferencias. Dentro de este tipo, se puede diferenciar un agrupamiento exclusivo, donde los datos pertenencen a un unico grupo, y un agrupamiento superpuesto, donde los datos pueden perteneces a varias agrupaciones. El ejemplo de las fotografías entra dentro de esta categoría.
	\item \textbf{Reglas de asociación}: utiliza una medida de interés para obtener un conjunto de reglas sólidas que permitan descubrir asociaciones interesantes entre las características de un conjunto de datos. La principal aplicación es el \guillemetleft análisis de cestas de compra\guillemetright, que se usa para determinar los patrones de compra de los clientes en funcion de las relaciones entre productos.
	\item \textbf{Reducción de dimensionalidad}: estos algoritmos buscan reducir la complejidad de un conjunto de datos de alta dimensión a espacios de baja dimensión sin perder propiedades fundamentales de los datos originales. Este tipo de algoritmos se utiliza en la fase de análisis de datos, facilitando la representación gráfica. Se puede ver un ejemplo a continuación.
\end{itemize}

%TODO: ORGANIZAR ESTA PARTE

\imagen{../img/memoria/ML-ReduceDimension.png}{Reducción de dimensionalidad}{1} 


%https://www.wolfram.com/language/introduction-machine-learning/machine-learning-par%adigms/img/2-machine-learning-paradigms-Print-6.en.png

En la siguiente tabla se resumen las principales diferencias entre aprendizaje supervisado y no supervisado:
\begin{table}[ht]
	\centering
	\begin{tabular}{@{}p{2.5cm} p{5cm} p{5cm}@{}}
		\toprule
			 & \textbf{Supervisado} & \textbf{No supervisado} \\
		\midrule
		\textbf{Objetivo} & Aproximar una función que asigna entradas a salidas a partir de un conjunto de datos clasificados. & Crear una representación concisa de los datos, posibilitando la generación de contenido creativo a partir de ellos. \\
		\addlinespace[0.5em]
		\textbf{Complejidad} & Complejidad simple. & Complejidad computacional mayor.\\
		\addlinespace[0.5em]
		\textbf{Entrada} & Se conoce el número de clases (datos etiquetados). & No se conoce el número de clases (datos no etiquetados). \\
		\addlinespace[0.5em]
		\textbf{Salida} & Genera un valor de salida esperado. & No se tienen valores de salida asociados \\
		\addlinespace[0.5em]
		\textbf{Tipos} & Clasificación, Regresión & Clustering, Reglas de asociación, Reducción de dimensionalidad \\
		\bottomrule
	\end{tabular}
	\caption{Comparación aprendizaje supervisado y no supervisado ~\cite{USL:guide}.}
	\label{supervisado_VS_noSupervisado}
\end{table}
\newpage


\section{Aprendizaje semi-supervisado}
Como el nombre sugiere, el aprendizaje semi-supervisado se encuentra entre los dos tipos vistos anteriormente. Los algoritmos dentro de esta estrategia se basan en extender cualquiera de los aprendizajes, supervisado o no supervisado, para añadir información adicional que el otro no proporciona ~\cite{Intro:SemiSupervised}.

Los métodos de clasificación semi-supervisada intentan utilizar puntos de datos no etiquetados para generar un modelo cuyo rendimiento supere el de los modelos obtenidos al utilizar solo datos etiquetados ~\cite{Engelen:semi-supervised}. \\Por ejemplo, imaginemos que se esta trabajando en la clasificación de imágenes médicas para identificar diferentes tipos de enfermedades. En este caso, consideramos específicamente la detección temprana de ciertos tipos de cáncer a partir de imágenes de tomografías. En un enfoque supervisado, podríamos entrenar un modelo utilizando un conjunto de datos etiquetado que incluye imágenes con diagnósticos de cáncer y sin cáncer. Sin embargo, la obtención de un gran conjunto de datos etiquetado puede ser costosa y consume tiempo. En un escenario de aprendizaje semi-supervisado, además de los datos etiquetados, podríamos tener un conjunto de datos mucho más grande que incluye imágenes no etiquetadas. Algunas de estas pueden contener señales sutiles o características asociadas con el cáncer que no han sido previamente etiquetadas.\\El modelo de aprendizaje semi-supervisado  podría analizar estas imágenes no etiquetadas y descubrir patrones que podrian indicar la presencia temprana de cáncer. Por ejemplo, podría aprender a reconocer características microscopicas especificas de las imagenes que no son evidentes para el ojo humano. Cuando se encuentra con nuevas imágenes no etiquetadas que comparten estas características, el modelo podría clasificarlas como indicativas de la presencia de cáncer, incluso si no ha visto exactamente esas características en el conjunto de datos etiquetado.
Existe una condición necesaria en el aprendizaje semi-supervisado: la distribución marginal subyacente $p(x)$ sobre el espacio de entrada debe contener información acerca de la distribución posterior $p(x|y)$ ~\cite{Engelen:semi-supervised}. Es decir, la naturaleza de los datos no etiquetados debe contener información útil para inferir las etiquetas correspondientes.
Esta suposición es básica y en la mayoría de los ejemplos se cumple. Aún asi, como la manera de interactuar entre $p(x)$ y  $p(x|y)$ no es siempre la misma, se pueden tomar malas decisiones que conllevarían un rendimiento cada vez peor. Por esta razón, existen tres principales suposiciones que todo algoritmo semi-supervisado debe cumplir para funcionar correctamente.
\begin{itemize}
	\item \textit{\textbf{Smoothness assumption}}: traducida como suposición de suavidad, consiste en que para dos puntos $x_{1}$ y $x_{2}$ que están cerca en una región densa, entonces sus correspodientes salidas (o etiquetas) $y_{1}$ y $y_{2}$ deben ser las mismas. Esto es útil sobretodo con datos no eitquetados, ya que por la propiedad transitiva, dos puntos que no estén relativamente cerca, pueden ser de la misma clase.
	\item \textit{\textbf{Low-density assumption}}: esta suposición está definida sobre la distribución de datos de entrada $p(x)$ y dice que el límite de decisión en la clasificación debe pasar antes por un área de poca densidad que por una de mayor densidad. Esto se puede observar en la figura \ref{fig:../img/memoria/Smoothness-LowDensity.png}.

	\imagen{../img/memoria/Smoothness-LowDensity.png}{\textit{Smoothness assumption} y \textit{Low-density assumption}~\cite{Engelen:semi-supervised}}{0.5}
	
	\item \textit{\textbf{Manifold assumption}}: esta suposición afirma que los datos utilizados se encuentran en un \textit{manifold} de baja dimensión incrustado en un espacio de mayor dimensión. En otras palabras, los datos, en lugar de proceder de cualquier parte del espacio, deben proceder de estos \textit{manifolds} de dimensiones más bajas ~\cite{web:assumptions}.
	\imagen{../img/memoria/ManifoldAssumption.png}{\textit{Manifold assumption}~\cite{web:assumptions}}{0.45}
\end{itemize}

En algunas ocasiones, aparece una cuarta suposición: \textit{\textbf{cluster assumption}}. Esta indica que dos datos que pertenecen a un mismo \textit{cluster}, pertenecen también a la misma clase. Se tomará esta suposición como una generalización de las tres anteriores ~\cite{Engelen:semi-supervised}.

\imagen{../img/memoria/ClusterAssumption.png}{\textit{Cluster assumption}~\cite{web:assumptions}}{1}.

No hay una clasificación oficial de algoritmos de aprendizaje semi-supervisado, pero si se pueden encontrar aproximaciones teniendo en cuenta las suposiciones en las que estan basadas los algoritmos y en como se relacionan con los algoritmos supervisados y no supervisados.
\imagen{../img/memoria/Clasificacion-SemiSupervised}{Clasificación de los diferentes algoritmos que pretenden incorporar datos no etiquetados a métodos de clasificación. Basado en ~\cite{Engelen:semi-supervised}}{1}.

\subsection{Métodos inductivos}
Los métodos inductivos pretenden construir un clasificador que pueda generar predicciones para cualquier objeto del espacio de entrada. En el entrenamiento de este clasificador o modelo se pueden utilizar datos no etiquetados, pero las predicciones cuando hay varios son independientes entre sí una vez finalizado el entrenamiento.
\begin{itemize}
	\item \textbf{\textit{Wrapper methods}}: Estos métodos entrenan inicialmente clasificadores con datos etiquetados y luego utilizan las predicciones para generar datos adicionales etiquetados. Los clasificadores se vuelven a entrenar con estos datos pseudo-etiquetados.
	\item \textbf{\textit{Unsupervised preprocessing}}: Estos métodos extraen características útiles, pre-agrupan datos o determinan parámetros iniciales de aprendizaje de manera no supervisada, pero solo se aplican a datos originalmente etiquetados. Mejoran el rendimiento de clasificadores supervisados al utilizar información de datos no etiquetados durante la etapa de preprocesamiento.
	\item \textbf{\textit{Intrinsecally semi-supervised}}: Incorporan directamente datos no etiquetados en la función objetivo o procedimiento de optimización. Son extensiones de métodos supervisados al entorno semi-supervisado. Maximizan la información obtenida de datos no etiquetados durante el proceso de aprendizaje.
	
\end{itemize}
\subsection{Métodos transductivos}
A diferencia de los métodos inductivos, los métodos transductivos no construyen un clasificador para todo el espacio de entrada. En su lugar, su poder predictivo se limita exactamente a los objetos que encuentra durante la fase de entrenamiento. Por lo tanto, los métodos transductivos no tienen fases de entrenamiento y prueba distintas.
Los métodos transductivos suelen definir un grafo sobre todos los puntos de datos, etiquetados y no etiquetados, codificando la similitud entre pares de puntos de datos con aristas posiblemente ponderadas. Posteriormente se profundizará en este ámbito.

\subsection{Ensembles}
\cite{ensembles}
Boosting, bagging
\subsection{Grafos}
\cite{Engelen:semi-supervised}


\section{\textit{Random Forest}}
/// Ejemplo de estructura
\section{\textit{Adaboost}}

\section{Diseño web}





\capitulo{4}{Técnicas y herramientas}

En este apartado se tratará de presentar las técnicas llevadas a cabo para desarrollar el proyecto y también las herramientas utilizadas durante todo el proceso.

\section{SCRUM}
Para explicar esta sección se utilizará el manual de la certificación oficial Scrum Master de Scrum Manager, ~\cite{SCRUM}.

\imagen{../img/memoria/cicloSCRUM.png}{Procedimiento en la metodología SCRUM ~\cite{SCRUM}}{1.1}.

La metodología ágil Scrum se diferencia de las metodologías clásicas en su enfoque iterativo y flexible hacia la gestión de proyectos. Mientras que las metodologías clásicas, como el modelo en cascada, siguen un enfoque secuencial y predeterminado, Scrum promueve la adaptabilidad y la colaboración continua. Facilita respuestas rápidas a los cambios a través de ciclos cortos de desarrollo llamados Sprints, permitiendo a los equipos evaluar el progreso y ajustar el rumbo con frecuencia.
\subsection{Roles}
\begin{itemize}
	\item \textbf{Desarrolladores}: Encargados de crear el producto, los desarrolladores son fundamentales para la ejecución técnica del proyecto, aportando habilidades específicas para alcanzar los objetivos del Sprint.
	\item \textbf{Propietario del producto(\textit{Product Owner})}: Define el alcance y las prioridades del proyecto, manteniendo la pila del producto actualizada para reflejar las necesidades del negocio, asegurando que el trabajo del equipo de desarrollo aporte el máximo valor.
	\item \textbf{\textit{Scrum Master}}: Facilitador y guía del equipo Scrum, el Scrum Master ayuda a implementar Scrum, asegurándose de que se sigan las prácticas y procesos, y trabaja para eliminar obstáculos que puedan impedir el progreso del equipo.
\end{itemize}
\subsection{Artefactos}
\begin{itemize}
	\item \textbf{Pila del producto(\textit{Product Backlog})}: Lista ordenada de todo lo necesario para el producto, gestionada por el \textit{Product Owner}, que establece los requisitos y prioridades.
	\item \textbf{Pila del sprint(\textit{Sprint Backlog})}: Conjunto de elementos seleccionados de la pila del producto para ser desarrollados durante el sprint, junto con un plan para entregar el incremento del producto y lograr el objetivo del Sprint.
	\item \textbf{Incremento}: La suma de todos los elementos de la pila del producto completados durante un sprint y todos los sprints anteriores, que cumple con los criterios de aceptación y asegura que el producto es potencialmente entregable.
\end{itemize}

\subsection{Eventos}
\begin{itemize}
	\item \textbf{\textit{Sprint}}: un periodo fijo durante el cual se crea un incremento del producto potencialmente entregable. Suele ser entre una y cuatro semanas.
	\item \textbf{Planificación del sprint}: sesion al inicio del sprint donde el equipo selecciona trabajo de la pila del producto para completar durante el sprint.
	\item \textbf{Reunión diaria}: breve reunion diaria para sincronizar actividades y crear un plan para el proximo dia, facilitando la colaboración.
	\item \textbf{Revisión del sprint}: al final del sprint, el equipo presenta el incremento a los interesados, recopilando retroalimentación para futuras iteraciones.
	\item \textbf{Retrospectiva del sprint}: oportunidad para el equipo scrum de inspeccionarse a sí mismo y crear un plan de mejoras para el proximo sprint.
\end{itemize}

\section{Herramientas}
Taiga, visual studio code, zappier, 
\capitulo{5}{Aspectos relevantes del desarrollo del proyecto}
En la sección que sigue, se abordan los aspectos más significativos que han marcado el desarrollo de este proyecto. Se detalla cómo cada elección ha influido en la trayectoria y los resultados del proyecto.
\section{Lectura artículos científicos}
Se ha decidido incluir la lectura de artículos científicos como aspectos relevantes ya que son una parte significativa de todo el trabajo, llevando bastante tiempo en aquellos que son más técnicos. Además proporcionan una base sólida de teorías y métodos que pueden ser utilizadas tanto para el trabajo como para otras situaciones. Todos ellos están en inglés por lo que también se mejora esta habilidad, notando cierta mejora de comprensión en los últimos artículos.
Un artículo científico es un informe escrito y publicado en una revista con cierto prestigio que describe resultados originales de investigación. Estos artículos son fundamentales para la difusión del conocimiento científico y suelen seguir un formato estructurado que incluye una introducción, metodología, resultados, discusión y conclusiones.

Para este trabajo se han leido 2 tipos de artículos, los \textit{surveys} o resúmenes sobre un tema y los artículos de implementación de algoritmos:
\begin{itemize}
	\item \textbf{\textit{A Survey on Semi-Supervised Learning}}~\cite{Engelen:semi-supervised}: Este artículo proporciona una revisión exhaustiva del campo del aprendizaje semi-supervisado. Gracias a este artículo se consigue una base sólida en este tipo de aprendizaje y se comprenden cuales son los desafios actuales y las direcciones futuras en la investigación en este campo.
	\item \textbf{\textit{Ensemble Based Systems in Decision Making}}~\cite{ensembles}: Este artículo revisa el uso de sistemas basados en \textit{ensembles} para la toma de decisiones. Estos sistemas combinan múltiples clasificadores para mejorar la precisión y la robustez del proceso de decisión. Destaca cómo los sistemas de ensamblaje pueden superar las limitaciones de los clasificadores individuales y proporcionar decisiones más confiables y precisas.
	\item \textbf{\textit{Improve Computer-Aided Diagnosis With Machine Learning Techniques Using Undiagnosed Samples}}~\cite{IEEE:CoForest}: Los autores presentan el método Co-Forest, una técnica semi-supervisada que combina múltiples clasificadores de árboles de decisión para mejorar la precisión del diagnóstico. Se discuten los beneficios de incorporar datos no etiquetados en el proceso de entrenamiento y se presentan resultados experimentales que demuestran la eficacia de esta metodología en varias aplicaciones médicas.
	\item \textbf{\textit{Graph Construction Based on Labeled Instances for Semi-supervised Learning}}~\cite{gbili}: Los autores presentan un método para la construcción de grafos basado en instancias etiquetadas para el aprendizaje semi-supervisado. La idea principal es utilizar la información de las instancias etiquetadas para guiar la construcción del grafo.
	\item \textbf{\textit{Learning with Local and Global Consistency}}~\cite{LGC}: Este artículo introduce un algoritmo para el aprendizaje con consistencia local y global.Los autores describen cómo el algoritmo itera para ajustar las etiquetas de las instancias no etiquetadas, combinando la información de las instancias vecinas y la información inicial. Se demuestra que este enfoque es eficaz para tareas de clasificación semi-supervisada.
\end{itemize}
\section{Trabajo Preexistente}
La elección de este trabajo se realiza por el gran interés en la inteligencia artificial y el aprendizaje automático, pero también por el hecho de que ya existía un trabajo realizado por otro alumno un año atrás (David Martínez Acha -- \url{https://vass.dmacha.dev/}). Esto ayudaría mucho en el desarrollo ya que serviría como referencia para muchas dudas.

El plan original era hacer mi propia página web educativa desde cero, pero mostrando otra serie de algoritmos (ensembles y grafos) en lugar de los ya existentes. Con el desarrollo del primer algoritmo surge la idea por parte del tutor de basar el proyecto en esta otra página desarrollada por David Martínez. De esta manera, el tiempo que hubiera empleado en aprender e implementar la web desde cero, se emplea en comprender todo el código programado por David, aprovechando también la gestión de cuentas de usuarios. Aún así, se deja total libertad para cambiar e implementar lo que haga falta para mejorar el proyecto original.

El tiempo empleado en ajustarse al nuevo código fue de dos sprints, ya que no solo trataba de leer código, sino de comprender las técnicas de HTML, css y javascript que se utilizan, junto con bibliotecas como \textit{Bootstrap}.
Aún así, el tiempo ganado es considerable y da pie a poder implementar más algoritmos y pensar en ideas que mejoran la web.
Inicialmente se piensa que con un fork a su repositorio de GitHub,~\cite{GH:VASS}, se puede trabajar mejor, pero de esta manera se perderían las tareas y commits hechos hasta la fecha en el repositorio de este proyecto. Por esto se decide descargar el contenido y copiarlo a el proyecto ya en desarrollo.

La documentación de David~\cite{TFG:David}, sirve de gran ayuda y también se heredan partes de ella, como los trabajos relacionados y conceptos teóricos. También se tiene en cuenta las líneas de trabajo futuras desarrolladas para poder implementarlas en este trabajo.
\subsubsection{Cambios en la web}
Gran parte de la web se reutiliza, pero a medida que se desarrolla el proyecto surgen nuevas ideas que modifican parte del comportamiento de la web que ya existía. Esta sección servirá para remarcar esas pequeñas o grandes modificaciones que sufre el diseño inicial y cual es la idea de esta nueva funcionalidad. 
La primera idea de cambio que surge es la de modificar la funcionalidad de configuración de un algoritmo. Visualizando páginas web, se da con una página que muestra algoritmos de aprendizaje automático~ \cite{web:ml-visualizer}, pero no de la misma manera que David. En esta página se permite configurar y ver resultados y estadísticas en una única ventana, con la gran diferencia de que no muetra el paso a paso en cada algoritmo, sino directamente el resultado final de la clasificación.
El primer prototipo se hace pensando en esta idea,quedando algo parecido a:
\imagen{../img/memoria/prototipo_coforest.png}{Prototipo de visualizacion de algoritmo Co-Forest}{1}
Posteriormente, gracias al tutor nos damos cuenta de que si la idea es mostrar el paso a paso, que los parametros de configuración estén en la misma pantalla, no es de gran ayuda. Por esto, se decide volver a la versión del trabajo base y modificar las plantillas necesarias para adaptarlas a los nuevos algoritmos.

\textbf{Configuración de parámetros:} Cuando tratamos con archivos de datos preparados para estos algoritmos, siempre se suele dar el caso de que la clase o etiqueta suele ser la última columna. Por esto, se ha mantenido la opción de seleccionar el atributo deseado, pero saldrá por defecto la clase en la caja de entrada. Esto además permite mayor agilidad a la hora de hacer pruebas de visualización ya que no hay que gastar tiempo en seleccionar esta opción.

\textbf{Gráfica de estadísticas específicas:} el cambio que se ha realizado en esta parte ha sido la opción de marcar o desmarcar todos los clasificadores para ver su traza en la gráfica de estadísticas. Sirve sobretodo en el caso del Co-Forest por el hecho de que puede haber gran cantidad de arboles, y querer ver uno concreto con la configuración anterior llevaba una perdida de tiempo innecesaria desmarcando uno por uno el resto de clasificadores.

\textbf{Utilizar fichero por defecto}: en la versión anterior se dedica una ventana entera a la selección del archivo, permitiendo descargar varios ficheros de prueba para luego subirlos. La idea aquí es agilizar el proceso de utilizar estos ficheros de prueba, aunque manteniendo la opción de descarga, permitiendo que al pulsar en un dataset de prueba pase a la fase de configuración directamente, sin tener que cargarlo desde las descargas.
\section{Algoritmos}
En esta sección se comentarán los aspectos más relevantes en la implementación de los algoritmos semisupervisados.
\subsection{Co-Forest}\label{sec5:coforest}
En la implementación de este algoritmo de nuevo se parte con una gran ventaja. El año anterior otra alumna había implementado el mismo algoritmo para otro proyecto. Dentro de esta aplicación, Patricia Hernando, hizo sus propios estudios, los cuales se han aprovechado en este trabajo.
Basado fuertemente en el pseudocodigo del artículo \cite{IEEE:CoForest}, la implementación se ve alterada por el parámetro W, el cual establece la confianza en las muestras para ser seleccionada o no. Resumiendo el estudio de Patricia, como se puede ver en el pseudocodigo del articulo, hay una ecuación donde el valor de W esta dividiendo. Esto es un problema ya que según establece el algoritmo puede llegar a valer cero, provocando así una indeterminación. Uno de los estudios de Patricia determina que una de las mejores soluciones a esto es iniciar el parámetro W al minimo entre 100 y el 10\% de la cantidad de muestras etiquetadas que hay. Como se puede ver en el pseudocodigo de la sección tres, esto se aprovecha, evitando así posibles problemas.
Para determinar si el algoritmo definitivo es bueno, se compara con el de Patricia, evaluando como varía el valor de \textit{accuracy} en cada iteración del algoritmo. Los resultados se muestran en la figura \ref{fig:../img/memoria/ComparacionCoForest.png}
\imagenflotante{../img/memoria/ComparacionCoForest.png}{Comparación algoritmo CoForest utilizando diversos datasets}{1}

\subsection{Grafos}
En este apartado se comentarán todos los aspectos relevantes relacionados con la implementación de los algoritmos basados en grafos. Desde sus posibles interpretaciones y modificaciones con respecto al código original, a los diferentes estudios de comparación.

\subsubsection{Comparativa de bibliotecas de grafos}
Cuando se quiere implementar un algoritmo basado en grafos, lo ideal es utilizar una biblioteca que ayude a automatizar y mejorar el código. En \textit{Python}, existen varias bibliotecas que ayudan en esta tarea, tres de ellas son: \textit{\textbf{NetworkX}}, \textit{\textbf{igraph}} y \textit{\textbf{graph-tool}}. En esta sección se resumirá el estudio realizado para elegir la opción que mejor se adapte a las especificaciones.

Todas ellas ayudan en la construcción de grafos, pero para empezar es necesario dejar claro para que se va a utilizar esta biblioteca. En cuanto al tamaño de los grafos, no necesariamente se necesita algo que maneje grafos muy grandes (más de 10000 nodos) de manera efectiva. La mayoría de \textit{datasets} utilizados tendrán muchas menos entradas de datos. En cuanto a la velocidad, se busca algo que sea efectivo pero sin necesidad de buscar lo mejor o más rápido, ya que los datos de entrada no van a suponer un gran esfuerzo. También hay que tener en cuenta la integración que se llevará a cabo posteriormente en la web, posiblemente con herramientas como \textit{d3.js}.

Tras una primera búsqueda queda claro que la herramienta de \textit{graph-tool} tiene un objetivo mucho más amplio y está pensado para proyectos con grafos grandes. De hecho es una herramienta que no se instala con \textit{pip} sino que necesita otra instalación.

Por lo tanto, descartada una opción, se realizará una pequeña prueba para llegar a una conclusión. A continuación se muestra el pseudocódigo utilizado para la comparación de herramientas.
\clearpage
\begin{algorithm}
	\label{testGraph}
	\KwIn{Dataset de prueba $L$ \textit{(digits)}}
	\KwOut{Grafo $G$ en formato \textit{JSON}}
	\BlankLine
	\textit{timer} $\leftarrow$ \textit{startTimer}()\\
	$G \leftarrow \emptyset$\\
	$D \leftarrow$ \textit{pairwiseDistances}($L$) // Matriz de distancias\\
	\For{$i = 0$ \KwTo $|L|-1$}{
		// Agregar vértices al grafo para cada muestra de $L$\\
		$G \leftarrow$ \textit{addNode}($i$)\\
	}
	$k \leftarrow 5$\\
	\For{$i = 0$ \KwTo $|L|-1$}{
		$kNN \leftarrow$ \textit{getKNearestNeighbors}($D[i]$, $k$)\\
		\For{$j \in kNN$}{
			// Añadir arista entre $i$ y $j$ con el peso de la distancia\\
			$G \leftarrow$ \textit{addEdge}($i$, $j$, $D[i][j]$)\\
		}
	}
	\textit{timer} $\leftarrow$ \textit{stopTimer}()\\
	\textit{print}("Tiempo de construcción y kNN: ", \textit{timer})\\
	$JSONGraph \leftarrow$ \textit{convertToJSON}($G$)\\
	\Return{$G$}
	\caption{\textit{NetworkX vs igraph}}
\end{algorithm}

Lo que se ha querido representar es el tiempo que tarda en construir el grafo y calcular los $k$ vecinos más cercanos \textit{(kNN)} para cada nodo. A su vez también se ha estudiado cuánta facilidad existe a la hora de convertir el grafo a formato \textit{JSON}, para que pueda ser procesado después por herramientas como \textit{d3.js}.

Los resultados obtenidos han sido los siguientes:
\begin{verbatim}
	Ejecución 1:
	NetworkX: Construction and kNN time: 0.0044 seconds
	igraph: Construction and kNN time: 0.0134 seconds
	
	Ejecución 2:
	NetworkX: Construction and kNN time: 0.0020 seconds
	igraph: Construction and kNN time: 0.0111 seconds
\end{verbatim}

A su vez, en el uso del formato \textit{JSON} se encuentra más útil el uso de NetworkX ya que incluye un método propio de exportación (\textit{nx.readwrite.json\_graph}).
Por el contrario, con \textit{igraph}, habría que constuir un diccionario recorriendo los nodos y enlaces y posteriormente pasarlo a formato \textit{JSON}.

En conclusión, se va a utilizar la libería \textit{\textbf{NetworkX}} por las ligeras mejoras en las pruebas realizadas y porque la curva de aprendizaje es algo menor que en el resto de casos \footnote{Finalmente se utiliza únicamente para representar gráficamente el grafo construido en cada paso, ya que mediante estructuras básicas como diccionarios o listas es posible implementar los algoritmos}.

\subsubsection{Modificaciones GBILI}\label{sec5:gbili}
Con respecto al algoritmo \textit{GBILI}, surgen varias complicaciones o situaciones que no dejan claro el funcionamiento correcto. En este apartado se comentan las que se creen son más relevantes.
\begin{enumerate}
	\item Visualización por pasos en la web: desde un principio se tiene claro que lo ideal es tener una visualización por pasos del grafo, localizando las principales fases y mostrando como van cambiando hasta llegar a la inferencia. En este caso, el pseudocódigo \ref{alg:Gbili} muestra como hay un bucle principal en el que cada lista se va construyendo para cada nodo recorrido en este bucle externo. Para la visualización requerida esto no es lo ideal, ya que no podemos aislar las principales fases: lista de vecinos, lista de vecinos mutuos, grafo de distancias mínimas y grafo definitivo. Por todo esto, se decide seguir la misma estructura pero implementando todas estas estructuras de forma consecutiva.
	\item Grafo no dirigido: en todo momento durante la implementación se conoce que cualquier enlace dentro del grafo es bidireccional, es decir, el grafo es no dirigido. Pero en una de las implementaciones esto no se sigue, lo que da en una estructura dirigida, que al realizar el seguimiento del código produce situaciones erróneas y confusas.
	\item Malinterpretaciones pseudocódigo: en el código del artículo pueden surgir ciertas dudas de como hace algún paso. Estas son: al hacer la búsqueda en anchura, dice que debe retornar la componente (subgrafo aislado) del grafo completo, posteriormente, esta sintaxis la utiliza para los nodos individuales. Después de consultarlo con el tutor se llega a la conclusión de que la búsqueda en anchura realmente devuelve todo el conjunto de componentes del grafo y posteriormente se accede a la que pertenece cada nodo individual. Otra sintaxis que puede llevar a confusión es la que encontramos en la línea 20. Cuando comprueba que la distancia es mínima, la suma la hace dentro de los dos bucles, lo que puede llevar a interpretación de que esa condición se debe cumplir dentro del bucle interno. En esta situación, pongamos el siguiente ejemplo: la lista actual de vecinos mutuos es {0: [1, 2], 1: [0]...} y dos de los nodos etiquetados son [3, 4...]. Se accede a la lista de vecinos del nodo 0 y despues se recorre toda la lista de nodos etiquetados. Viendo esto sabriamos que obtendriamos una distancia mínima hasta un nodo etiquetado, pero siempre se estaría guardando el enlace entre los nodos vecinos, obteniendo la misma estructura que la lista de vecinos mutuos. Por ello, al implementarlo, hay que tener en cuenta que esta condición debe ir fuera del bucle interno, para poder coger de verdad la mínima distancia entre un nodo y sus vecinos.
	\item Visualización con NetworkX: Para tener un primer acercamiento a una visualización y también para comprobar que el algoritmo ha seguido correctamente la influencia de los nodos etiquetados, se decide mostrar los 4 pasos graficamente. Un aspecto importante en la implementación es que con networkX, a la hora de construir el grafo, debe estar ordenado de la misma manera que los datos de entrada al algoritmo, si no mostrará información falsa, como nodos pintados como etiquetados que no lo son.
	\imagen{../img/memoria/networkxGBILI.png}{Visualización correcta de las 4 fases de construcción del grafo con el algoritmo GBILI utilizando el \textit{dataset} de \textit{iris data}}{1}
\end{enumerate}
\subsubsection{LGC adaptado a grafos}\label{sec5:LGC}
En el algoritmo original de Consistencia Local y Global, la matriz de afinidad $W$ se construye utilizando una función exponencial sobre la matriz de distancias entre los puntos. Sin embargo, dado que en este caso ya se ha construido un grafo $G$, utilizaremos esta información para definir la matriz de afinidad $W$. Específicamente, $W$ será una matriz binaria donde $W_{ij}=1$ si hay un enlace entre los nodos $i$ y $j$ en el grafo $G$, y $W_{ij}=0$ de lo contrario. Esta modificación aprovecha la estructura del grafo previamente construida, además, es la misma seguida por loa autores de~\cite{gbili}. Aunque es una opción algo "brusca", debe funcionar cuando los parámetros de los algoritmos son lo suficientemente buenos.

Este ajuste del algoritmo de Consistencia Local y Global permite aprovechar la estructura del grafo construido a partir de los puntos de datos, en lugar de basarse únicamente en la matriz de distancias. Esta modificación es especialmente útil cuando se tiene información topológica adicional, mejorando potencialmente la precisión en la inferencia de etiquetas para los puntos no etiquetados.

Un apunte importante es que cuando se estaba implementando esta modificación, se piensa que la matriz de afinidad se debe de construir aplicando la misma fórmula que antes, pero tomando como matriz de distancias las nuevas medidas de 0s y 1s. Esto no debe ser así y fue corregido ya que si no la matriz de afinidad ya no sería esa matriz binaria que utilizan los autores del artículo original.

Para saber si funciona correctamente este algoritmo se realiza un pequeño estudio variando 3 parámetros, para ver como varía la \textit{accuracy} y que parámetro es el que afecta más al algoritmo.
Los conjuntos de datos utilizados en este estudio son: \textit{iris}, \textit{breast cancer} y \textit{wine}.

Cada conjunto de datos se divide en una parte de entrenamiento (etiquetada) y una parte de prueba (no etiquetada). Los parámetros a variar son:

\begin{itemize}
	\item Número de vecinos más cercanos \(K\): 5, 7, 10
	\item Parámetro de suavizado \(\alpha\): 0.69, 0.79, 0.99
	\item Tolerancia para la convergencia: \(1e-6\), \(1e-5\), \(1e-4\)
\end{itemize}

Para cada combinación de parámetros, se construye un grafo utilizando el algoritmo GBILI. Posteriormente, se aplica el algoritmo LGC para inferir etiquetas en el grafo, utilizando los parámetros especificados.

Para un análisis estadístico, se utiliza el análisis ANOVA (Análisis de Varianza) para evaluar la influencia de cada parámetro en la precisión de las predicciones. ANOVA (Análisis de Varianza) es una técnica estadística utilizada para comparar las medias de tres o más grupos y determinar si al menos una de las medias es significativamente diferente de las demás, \cite{web:anova}. En este estudio, ANOVA se utiliza para evaluar la influencia de los parámetros \(K\), \(\alpha\) y la tolerancia en la precisión de las predicciones.

\begin{itemize}
	\item \textbf{Sum of Squares (SumSq)}: Mide la variabilidad total de la precisión debido a cada parámetro.
	\item \textbf{Degrees of Freedom (df)}: Representa el número de valores independientes que pueden variar.
	\item \textbf{F-Statistic (F)}: Compara la variabilidad entre los grupos con la variabilidad dentro de los grupos.
	\item \textbf{P-Value (PR(>F))}: Indica la probabilidad de observar un efecto tan extremo como el observado si la hipótesis nula es cierta. Un valor pequeño sugiere que el parámetro tiene un efecto significativo.
\end{itemize}

Los resultados del análisis ANOVA muestran que el número de vecinos más cercanos \(K\) tiene una influencia significativa en la precisión del modelo, con un valor de \(p\) muy pequeño (\(1.19e-18\)). El parámetro \(\alpha\) y la tolerancia no muestran una influencia significativa en la precisión, con valores de \(p\) de 0.890 y 1.000 respectivamente. 
\imagen{../img/memoria/estudioGBILI-LGC.png}{Efecto del parámetro K sobre la precisión}{1}
Además, basado en los resultados del estudio, los mejores parámetros por defecto son:
\begin{itemize}
	\item \(K = 7\)
	\item \(\alpha = 0.79\)
	\item Tolerancia = \(1e-06\)
\end{itemize}

Estos parámetros proporcionan la mayor precisión promedio en la inferencia de etiquetas no supervisadas.
Hay que tener en cuenta que es un estudio bastante pequeño y podría verse afectado por la falta de valores o incluso parámetros.

\capitulo{6}{Trabajos relacionados}
En esta sección, se analizarán los trabajos relacionados con el proyecto, destacando cómo cada uno de ellos ha influido y enriquecido el desarrollo. Es esencial reconocer y comprender estos trabajos, ya que ofrecen una base sobre la cual podemos construir, identifican oportunidades para la innovación y permiten evitar la repetición de errores pasados
\subsection{Visualizador de Algoritmos SemiSupervisados (VASS)}
Un pilar fundamental en el desarrollo de este proyecto es el trabajo realizado por David Martinez Acha. Este trabajo no solo ha sentado las bases conceptuales sino que también ha proporcionado una implementación práctica de gran valor. \textbf{Se hereda} directamente \textbf{todo el esfuerzo de David}, \textbf{incluyendo} sus investigaciones y \textbf{los trabajos relacionados} que identificó como relevantes en su estudio. Esta herencia es particularmente valiosa en lo que respecta a la implementación de la web, donde la infraestructura y el diseño preexistentes ofrecen una plataforma robusta desde la cual se puede avanzar sin partir de cero (ver figura~\ref{fig:../img/memoria/vass1.png}).

La implementación web desarrollada por David se destaca por su claridad, ofreciendo un punto de partida sólido para propias innovaciones. Aprovechar esta base preexistente permite enfocarse en la expansión y mejora de la funcionalidad, en lugar de invertir tiempo y recursos significativos en reconstruir lo que ya se ha logrado con éxito.

En resumen, la aplicación de David tiene dos grandes funcionalidades separadas: los algoritmos y la gestión de usuarios. Lo que realmente importa para este proyecto es la implementación web de la parte de los algoritmos. Se tiene una pantalla principal donde permite seleccinar el algoritmo, una posterior donde se introduce el dataset, una ventana de configuración de parámetros del algoritmo y por úlitmo la visualización de los resultados.\\
La idea es mantener las mismas funcionalidades en el mismo orden, pudiendo modificar o extender ciertas partes gráficas que unifiquen o lo hagan más sencillo.

\imagen{../img/memoria/vass1.png}{Página principal de la web \url{https://vass.dmacha.dev/}}{0.9}{Página inicial de VASS 1.0}
\subsection{ML Algorithm Visualizer}
Como ya se ha comentado en la sección anterior, esta página web fue importante en el planteamiento del primer diseño de la web. Se trata de una página web de visualización de algoritmos de aprendizaje automático, cuyo sitio web es \url{https://ml-visualizer.herokuapp.com/}. Con un diseño compacto permite seleccionar el modelo junto con la configuración de sus paramétros y unos segundos más tarde mostrará el resultado en la misma página, todo ello en una misma ventana como se puede ver en la imagen \ref{fig:../img/memoria/MLVisual-web.png}. Es aquí donde se diferencia del objetivo de este proyecto, donde se busca tener las ejecuciones separadas paso por paso.

\imagen{../img/memoria/MLVisual-web.png}{Captura de pantalla de la web \textit{ML Algorithm Visualizer}.}{0.9}{\textit{ML Algorithm Visualizer}}

\subsection{Clustering Visualizer}
Entre las muchas páginas web que existen de visualización de algoritmos, esta es de las pocas que se asemejan a una implementación de grafos, con sus nodos y sus enlaces. Permite visualizar diferentes algoritmos de clustering y análisis de datos, y, aunque no tenga mucha relación con los algoritmos implementados en este trabajo, la parte gráfica es interesante. En la figura \ref{fig:../img/memoria/clustering-visualizer.png} se observa una representación de nodos y enlaces clara, con diferentes colores para cada \textit{cluster}, lo que puede ser de gran ayuda a la hora de pensar la visualización de los grafos.
\imagen{../img/memoria/clustering-visualizer.png}{Ejecución de \textit{Kmeans} en la web \url{https://clustering-visualizer.web.app/}}{0.9}{Ejecución de \textit{K-Means} en \textit{Clustering Visualizer}}
\subsection{Aprendizaje semisupervisado en ciberseguridad}
Este trabajo corresponde con el Trabajo Fin de Grado de Patricia Hernando~\cite{TFG:Patricia}. Pese a ser una herramienta destinada a la ciberseguridad, tiene relación con el actual trabajo en el uso de algoritmos semisupervisados. Permite ver la implementación y la explicación del código del Co-Forest. Como ya se ha comentado, son de gran ayuda los estudios realizados para mejorar este algoritmo, ya que sirven para mostrar en la web una opción que permita seleccionar distintos valores y localizar diferencias en los resultados.

\subsection{Kaggle Notebooks}
\textit{Kaggle} es una plataforma en línea que permite a los profesionales de la ciencia de datos competir en desafíos de análisis de datos, compartir sus soluciones y colaborar en proyectos de aprendizaje automático.

Una de las características más destacadas de \textit{Kaggle} es su entorno de \textit{notebooks} interactivos, conocidos como \textit{Kaggle Notebooks}. Estos \textit{notebooks} permiten a los usuarios cargar y explorar datasets de manera intuitiva y visual. Cuando se carga un archivo de datos en un \textit{Kaggle Notebook}, su contenido se muestra automáticamente en forma de tabla, lo que facilita la inspección y comprensión de los datos (ver figura~\ref{fig:../img/memoria/kaggleNote.png}).

\imagen{../img/memoria/kaggleNote.png}{Visualización de datos en \url{https://www.kaggle.com/code}}{0.9}{Tabla de datos en \textit{Kaggle Notebook}}

Esta capacidad de visualización inmediata y eficiente ha inspirado el uso de tablas en este trabajo para la visualización de datos. Al mostrar los datos en formato tabular, es más sencillo identificar patrones, detectar anomalías y realizar análisis preliminares antes de proceder con las siguientes fases. 
\capitulo{7}{Conclusiones y Líneas de trabajo futuras}

En este apartado, se presentan las conclusiones más relevantes del proyecto, destacando los logros y los aprendizajes teniendo en cuenta los objetivos establecidos al inicio. Además, se proporciona un análisis crítico que sugiere posibles mejoras y propone líneas de trabajo futuras para seguir avanzando en esta área.

\section{Conclusiones}

En esta sección se presentan las conclusiones técnicas y personales derivadas del desarrollo del proyecto.

\subsection{Conclusiones Técnicas}

A lo largo de este proyecto, se han abordado y cumplido varios objetivos técnicos clave, en los que se puede sacar varias conclusiones por cada uno:

\begin{itemize}
	\item \textbf{Implementación de Algoritmos de Aprendizaje Semisupervisado}: Se han implementado varios algoritmos basados en artículos científicos, logrando buenos resultados en la clasificación y predicción de datos. La comparación con implementaciones de terceros ha permitido validar la eficacia de los algoritmos desarrollados.
	\item \textbf{Desarrollo Continuo de la Web}: Se ha continuado el desarrollo de la web ya implementada, mejorando su funcionamiento y añadiendo nuevas funcionalidades. Esto incluye la optimización de la interfaz de usuario y la incorporación de nuevas características basadas en las necesidades del usuario final.
	\item \textbf{Experimentación en Aprendizaje Automático}: Se han aprendido y aplicado métodos rigurosos para la experimentación en aprendizaje automático, asegurando resultados reproducibles y confiables. Esto incluye la preparación de datos, la validación cruzada y el análisis de métricas de desempeño.
	\item \textbf{Conocimiento de herramientas y técnicas}: sobretodo en el desarrollo web, con el que se había <<jugado>> muy poco en la carrera, se ha logrado tener un buen conocimiento base de algunas herramientas comunes y de buenas prácticas en su uso.
	\item \textbf{Despliegue de Aplicaciones Web}: Se adquirieron conocimientos sobre el proceso completo de despliegue de una aplicación web, incluyendo la configuración de servidores, la implementación de servidores web y de aplicaciones, y la gestión de dominios y certificados de seguridad. 
\end{itemize}

\subsection{Conclusiones Personales}
En lo personal, siento que el trabajo realizado es bueno, pero que, siendo crítico, se podría haber hecho mejor en ciertos aspectos.

Hay que tener en cuenta que gran parte del trabajo en este tipo de proyectos parece <<invisible>>, como por ejemplo la lectura de artículos científicos o el hecho de depurar algoritmos, ya que difícilmente salen bien a la primera. Este tipo de tareas llevan un tiempo considerable que hay que tener en cuenta a la hora de evaluar el trabajo.

Siempre se escucha decir a profesores y profesionales que cuando heredas un trabajo que no has desarrollado tú mismo, muchas veces es más difícil empezar a implementar, en este caso no ha sido del todo así. Aunque estuve un tiempo considerable para conseguir navegar fluidamente por el proyecto, el hecho de que David hiciera un código mantenible y extensible, ayudó mucho en las nuevas funcionalidades. Aún así, creo que no he terminado de aprovechar esta ventaja de tener un proyecto base bien estructurado (todo sea dicho que la decisión de hacer una segunda versión no surge desde el inicio). Se podrían haber implementado más algoritmos, ya sea de grafos o de \textit{ensembles} y también se podría haber desarrollado algún paso más en el proceso seguido en la web.

En la web, me llevo un gran aprendizaje con respecto a los lenguajes, herramientas y técnicas que se utilizan comúnmente. El no tener conocimientos previos me limitó al inicio para poder entender y avanzar el código existente, pero finalmente se ha conseguido una base sólida. Al contrario pasa con el ámbito del aprendizaje automático, que, aunque se ha tenido que leer e investigar, se partía con una buena base vista en asignaturas como Sistemas Inteligentes, Computación Neuronal y Minería de Datos.

En conclusión, aunque se ha logrado cumplir con los objetivos principales del proyecto, siempre hay margen para mejorar. La experiencia adquirida en el despliegue de aplicaciones web y la implementación de algoritmos de aprendizaje automático ha sido invaluable. Sin embargo, es evidente que se podría haber optimizado el tiempo y los recursos para incluir más funcionalidades y optimizar aún más el proyecto.

\section{Líneas de trabajo futuras}
Hacer una aplicación web siempre conlleva mejoras sea del estilo que sea, siempre se pueden añadir funcionalidades o mejorar las que existen. Por eso este proyecto no va a ser menos, estando abierto a muchas opciones de cambio. El hecho de que el proyecto sea para la docencia deja aún más puertas abiertas, ya que el mundo de la inteligencia artificial/aprendizaje automático evoluciona a un ritmo muy alto. De nuevo se vuelven a contar con las líneas de futuro presentadas en el trabajo anterior,~\cite{TFG:David}, y que no se han implementado en este.

\begin{itemize}
	\item \textbf{Implementar nuevos algoritmos semisupervisados}. Como se ha visto en la sección~\ref{sec3}, existe una amplia clasificación de este tipo de algoritmos. Hasta ahora se puede considerar que existen métodos de envoltura (\textit{wrapper}), \textit{ensembles} y métodos basados en grafos. Si esta gama de tipos se amplia se podría hacer una web bastante llamativa con una selección inicial diferente a la de ahora, contando con un nivel más alto.
	\item \textbf{Permitir una carga de archivos más amplia}. Por el momento, si el usuario sube un archivo para su clasificación, pero este contiene atributos categóricos, el proceso no sigue adelante, avisando del error. Para permitir mayor flexibilidad al usuario, se podría implementar un sistema de lectura de archivos mejorado que procese estos datos y los transforme en atributos numéricos.
	\item \textbf{Visualizar comparaciones de resultados de ejecuciones}. Aprovechando que en el trabajo anterior se trabajó en una gestión de usuarios los cuales pueden almacenar las ejecuciones llevadas a cabo, implementar una ventana en la que se compare las métricas de cada uno o cómo ha clasificado cada uno un dato, podría ser interesante para el usuario.
	\item \textbf{Tutoriales en fase de configuración}. Un estudiante puede estar interesado en algunos de los algoritmos en concreto, y aún con las explicaciones y el pseudocódigo, puede que en la configuración del algoritmo no se entienda bien que hace cada parámetro. Añadir un tutorial o una guía de uso de cada parámetro en cada configuración aclararía los conceptos para poder ejecutar adecuadamente.
	\item \textbf{Seguimiento de los resultados}: Actualmente la aplicación cumple su objetivo de poder enseñar como funciona un algoritmo de aprendizaje semisupervisado, pero ya que realmente se están prediciendo y evaluando datos de entrada reales, se puede realizar un paso más en el que se vean los resultados finales de cada entrada de datos, viendo su valor real y su valor predicho. Además de poder asociarlos con las representaciones para tener situados a todos los datos.
	\item \textbf{Ciberseguridad en la web}: Hay muchos peligros a los que las páginas webs se exponen, sobretodo aquellas que tratan con datos sensibles, por ello estaría bien implementar más medidas de seguridad además de CSRF. Realmente esta línea de trabajo está pensada en el sentido de aprender las técnicas que existen de \textit{hacking} web y como evitarlas, ya que no es un objetivo real de ciberdelincuentes.
	\item \textbf{Detallar pasos de los grafos}: la visualización de los grafos resultó compleja y llevó gran parte del tiempo, pero estos algoritmos dan pie a detallar mucho más lo que está pasando por detrás del algoritmo. Por ejemplo, se podría indicar en todo momento cual es la lista de enlaces de un nodo seleccionado, junto con su distancia calculada, para entender porque se une a unos nodos y a otros no.
\end{itemize}


\bibliographystyle{plain}
\bibliography{bibliografia}

\end{document}
